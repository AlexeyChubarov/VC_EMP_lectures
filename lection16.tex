\chapter{}
\label{lecture16}
На прошлых лекциях мы убедились, что решение задачи о колебаниях мембраны требует знания собственных значений и собственных функций оператора Лапласа в области, форма которой равна форме мембраны при однородных граничных условиях. К сожалению, нахождение в явном виде и собственных значений, и собственных функций для мембраны произвольного вида невозможно. Однако, если мембрана прямоугольная или круглая, то собственные значения и собственные функции могут быть найдены методом Фурье.
\section{Случай прямоугольной мембраны.}
\label{lecture16section1}
Мембрана занимает область
\begin{equation*}
	\hfill G=\left\{x,y|0\leqslant x\leqslant l_1,\ 0\leqslant y\leqslant l_2\right\}_{\displaystyle.}\hfill
\end{equation*}




\tikzset{every picture/.style={line width=0.75pt}} %set default line width to 0.75pt        

\begin{tikzpicture}[x=0.75pt,y=0.75pt,yscale=-1,xscale=1]
	%uncomment if require: \path (0,236); %set diagram left start at 0, and has height of 236
	
	%Shape: Axis 2D [id:dp40755820928102016] 
	\draw  (36,191.27) -- (346.78,191.27)(67.08,15) -- (67.08,210.85) (339.78,186.27) -- (346.78,191.27) -- (339.78,196.27) (62.08,22) -- (67.08,15) -- (72.08,22)  ;
	%Shape: Rectangle [id:dp9678514260005915] 
	\draw   (67.08,103.85) -- (215.78,103.85) -- (215.78,191.27) -- (67.08,191.27) -- cycle ;
	
	% Text Node
	\draw (349,193.4) node [anchor=north west][inner sep=0.75pt]    {$x$};
	% Text Node
	\draw (46,6.4) node [anchor=north west][inner sep=0.75pt]    {$y$};
	% Text Node
	\draw (54,192.4) node [anchor=north west][inner sep=0.75pt]    {$0$};
	% Text Node
	\draw (214,192.4) node [anchor=north west][inner sep=0.75pt]    {$l_{1}$};
	% Text Node
	\draw (54,96.4) node [anchor=north west][inner sep=0.75pt]    {$l_{2}$};
	% Text Node
	\draw (69,172.4) node [anchor=north west][inner sep=0.75pt]    {$A$};
	% Text Node
	\draw (218,172.4) node [anchor=north west][inner sep=0.75pt]    {$B$};
	% Text Node
	\draw (218,86.4) node [anchor=north west][inner sep=0.75pt]    {$C$};
	% Text Node
	\draw (69,86.4) node [anchor=north west][inner sep=0.75pt]    {$D$};
	% Text Node
	\draw (130,137.4) node [anchor=north west][inner sep=0.75pt]    {$G$};
	
	
\end{tikzpicture}
\vspace{0.2cm}

Наша цель --- найти собственные функции и собственные значения оператора $-\Delta$ в области $G$ с однородными граничными условиями на $\partial G$. Эти условия в принципе могут быть свои на каждой границе. Для разнообразия рассмотрим случай, когда сторона $AD$ закреплена, а остальные --- свободны, тогда мы получим следующую задачу.
\begin{equation}\label{l16:eq:1}
	\hfill-\Delta v(x,y)=\lambda\cdot v(x,y),\hfill
\end{equation}  
\begin{equation}\label{l16:eq:2}
	\hfill v(0,y)=0,\quad y\in[0,l_2],\qquad \pder{v}{x}(l_1,y)=0,\quad y\in[0,l_2],\hfill
\end{equation}
\begin{equation}\label{l16:eq:3}
	\hfill \pder{v}{y}(x,0)=\pder{v}{y}(x,l_2)=0,\quad  x\in[0,l_1].\hfill
\end{equation}
Будем искать $v(x,y)=X(x)\cdot Y(y)$. Подставив в ~\eqref{l16:eq:1} и поделив на произведение $X\cdot Y$ получим 
\begin{equation}\label{l16:eq:4}
	\hfill-\frac{X''}{X}-\frac{Y''}{Y}=\lambda.\hfill
\end{equation}
Отношение $\dfrac{X''}{X}$ $\left\{\dfrac{Y''}{Y}\right\}$ не зависит от $y$ \{от $x$\}, значит и отношение $\dfrac{Y''}{Y}$ $\left\{\dfrac{X''}{X}\right\}$ не зависит от $y$ \{от $x$\}. Поэтому из~\eqref{l16:eq:4} следует, что каждое из этих отношений постоянно
\begin{equation}\label{l16:eq:5}
	\hfill-\frac{X''}{X}=\nu,\quad-\frac{Y''}{Y}=\mu\hfill
\end{equation} 
откуда
\begin{equation}\label{l16:eq:6}
	\hfill\text{a)}\ \ -X''=\nu\cdot X;\qquad\text{b)}\ \ -Y''=\mu\cdot Y.\hfill
\end{equation}

Прежде чем решать задачу дальше, заметим, что при выводе~\eqref{l16:eq:6},  форма мембраны (то есть вид области $G$) не имела значения. Но она будет играть решающую роль для получения граничных условий для $X(x)$ и $Y(y)$.

Подставим выражение $v=X(x)\cdot Y(y)$ в~\eqref{l16:eq:2},~\eqref{l16:eq:3}; получим
\begin{equation}\label{l16:eq:7}
	\hfill X(0)\cdot Y(y)=0,\quad X'(l_1)\cdot Y(y)=0,\qquad y\in[0,l_2],\hfill
\end{equation}
\begin{equation}\label{l16:eq:8}
	\hfill X(x)\cdot Y'(0)=0,\quad X(x)\cdot Y'(l_2)=0,\qquad x\in[0,l_1].\hfill
\end{equation}
Так как равенства~\eqref{l16:eq:7} выполняются для $\forall y\in[0,l_2]$, а~\eqref{l16:eq:8} для $\forall x\in[0,l_1]$, то из~\eqref{l16:eq:7} и~\eqref{l16:eq:8} следует, что
\begin{equation}\label{l16:eq:9}
	\hfill X(0)=0,\quad X'(l_1)=0,\hfill
\end{equation}
\begin{equation}\label{l16:eq:10}
	\hfill Y'(0)=0,\quad Y'(l_2)=0.\hfill
\end{equation}
Таким образом мы получили, что в силу~\eqref{l16:eq:6},~\eqref{l16:eq:9},~\eqref{l16:eq:10} функции $X(x)$ --- это собственные функции оператора $-\displaystyle\dder{}{x}$ и $\nu$ --- соответствующее собственное значение при граничных условиях~\eqref{l16:eq:9}, а $Y(y)$ и $\mu$ --- это собственные функции и собственные значения оператора $-\displaystyle\dder{}{y}$ с граничными условиями~\eqref{l16:eq:10}. Обозначим эти собственные значения и нормированные собственные функции соответственно через $\nu_k$, $X_k$ и $\mu_m$, $Y_m$\footnote{Я умышленно их не нахожу в явном виде: вы должны это уметь даже если заболеете коронавирусом.}. В силу~\eqref{l16:eq:4}
\begin{equation*}
	\hfill\lambda=\lambda_{km}=\nu_k+\mu_m\quad\text{и}\quad v(x,y)=v_{km}(x,y)=X_k\cdot Y_m.\hfill
\end{equation*}

Теперь можно бы вернуться к общей схеме решения задач о свободных или вынужденных колебаниях мембраны, заменяя в них индекс $n$ на пару $km$ (я об этом предупреждал). Однако предварительно надо доказать, что оператор $-\Delta$ с граничными условиями~\eqref{l16:eq:2},~\eqref{l16:eq:3} не имеет других собственных функций кроме $v_{km}$ и собственных значений кроме $\lambda_{km}$.
\begin{proof}
	Пусть $h(x,y)$ и $\gamma$ --- собственная функция и собственное значение оператора $-\Delta$ с граничными условиями~\eqref{l16:eq:2},~\eqref{l16:eq:3}. Пусть 
	\begin{equation*}
		\hfill h_{km}\eqdef\big(h, v_{km}\big)\quad\text{и}\quad\widehat{h}\eqdef h-\sum\limits_{k,m=1}^{\infty}h_{km}\cdot v_{km}.\hfill
	\end{equation*}
	Очевидно, что $\widehat{h}\perp v_{km}$. Фиксируем в функции $\widehat{h}(x,y)$ значение $y$ и по теореме Стеклова разложим функцию $\widehat{h}(x,y)$ как функцию от $x$ по собственным функциям $X_k$:
	\begin{equation}\label{l16:eq:11}
		\hfill\widehat{h}(x,y)=\sum\limits_{k=1}^{\infty}\alpha_k(y)\cdot X_k(x),\hfill
	\end{equation} 
	где $\displaystyle\alpha_k(y)=\smallint\limits_0^{l_1}\widehat{h}\cdot X_k(x)\,dx$. Разложим функцию $\alpha_k(y)$ по собственным функциям $Y_m(y)$. Тогда 
	\begin{equation*}
		\hfill\alpha_k=\sum\limits_{m=1}^{\infty}\beta_{km}\cdot Y_{m}(y),\hfill
	\end{equation*}
	где
	\begin{equation*}
		\beta_{km}=\int\limits_0^{l_2}\alpha_k(y)\cdot Y_{m}(y)\,dy=\int\limits_0^{l_2}\!\int\limits_0^{l_1}\widehat{h}(x,y)\cdot X_k(x)\cdot Y_m(y)\,dxdy=\int\limits_0^{l_2}\!\int\limits_0^{l_1}\widehat{h}(x,y)\cdot v_{km}(x,y)\,dxdy.
	\end{equation*}
	Но мы знаем, что $\widehat{h}\perp v_{km}$ по построению и, значит, $\beta_{km}=0$ и поэтому $\alpha_k=0$. В силу~\eqref{l16:eq:11} $\widehat{h}(x,y)=0$ и значит
	\begin{equation}\label{l16:eq:12}
		\hfill h=\sum\limits_{k,m=1}^{\infty}h_{km}\cdot v_{km}.\hfill
	\end{equation}
	Если для данных $k,m$ собственное значение $\gamma$ $\left(-\Delta h=\gamma\cdot h\right)$ не равно $\lambda_{km}$, то $h_{km}=(h,v_{km})=0$, ибо собственные функции оператора $-\Delta$, отвечающие различным собственным значениям, ортогональны. Поэтому в сумме~\eqref{l16:eq:12} останется лишь конечное число слагаемых (может --- одно)
	\begin{equation}\label{l16:eq:13}
		\hfill h=\sum\limits_{k,m}h_{km}\cdot v_{km},\quad \lambda_{km}=\gamma.\hfill
	\end{equation}
	Отсюда следует, что $\gamma$ обязательно есть одно из собственных значений $\lambda_{km}$, полученных методом разделения переменных (иначе $h\equiv0$) и что $h$ --- есть линейная комбинация функций $v_{km}$ из $\Ul[\lambda_{km}]$. Таким образом \emph{наше утверждение доказано}.\hfill\\
\end{proof}  
\section{Вынужденные колебания прямоугольной мембраны.}
\label{lecture16section2}
Если вынужденные колебания происходят только под действием распределённой силы, а граничные условия --- однородные, то такую задачу мы решать уже умеем (см.~лекцию 15). Здесь мы рассмотрим случай вынужденных колебаний при неоднородных граничных условиях. Пусть на мембрану действует распределённая сила с плотностью $\widetilde{f}(P,t)$ на единицу массы. Тогда уравнение колебаний мембраны 
\begin{equation}\label{l16:eq:14}
	\hfill\pdder{u}{t}=a^2\cdot\Delta u+\widetilde{f}(P,t).\hfill
\end{equation} 




\tikzset{every picture/.style={line width=0.75pt}} %set default line width to 0.75pt        

\begin{tikzpicture}[x=0.75pt,y=0.75pt,yscale=-1,xscale=1]
	%uncomment if require: \path (0,236); %set diagram left start at 0, and has height of 236
	
	%Shape: Axis 2D [id:dp40755820928102016] 
	\draw  (36,191.27) -- (346.78,191.27)(67.08,15) -- (67.08,210.85) (339.78,186.27) -- (346.78,191.27) -- (339.78,196.27) (62.08,22) -- (67.08,15) -- (72.08,22)  ;
	%Shape: Rectangle [id:dp9678514260005915] 
	\draw   (67.08,103.85) -- (215.78,103.85) -- (215.78,191.27) -- (67.08,191.27) -- cycle ;
	
	% Text Node
	\draw (349,193.4) node [anchor=north west][inner sep=0.75pt]    {$x$};
	% Text Node
	\draw (46,6.4) node [anchor=north west][inner sep=0.75pt]    {$y$};
	% Text Node
	\draw (54,192.4) node [anchor=north west][inner sep=0.75pt]    {$0$};
	% Text Node
	\draw (214,192.4) node [anchor=north west][inner sep=0.75pt]    {$l_{1}$};
	% Text Node
	\draw (54,96.4) node [anchor=north west][inner sep=0.75pt]    {$l_{2}$};
	% Text Node
	\draw (69,172.4) node [anchor=north west][inner sep=0.75pt]    {$A$};
	% Text Node
	\draw (218,172.4) node [anchor=north west][inner sep=0.75pt]    {$B$};
	% Text Node
	\draw (218,86.4) node [anchor=north west][inner sep=0.75pt]    {$C$};
	% Text Node
	\draw (69,86.4) node [anchor=north west][inner sep=0.75pt]    {$D$};
	% Text Node
	\draw (130,137.4) node [anchor=north west][inner sep=0.75pt]    {$G$};
	
	
\end{tikzpicture}
\vspace{0.2cm}

Предположим, что нам задан закон движения краёв $AB$ и $CD$ мембраны, а на краях $AD$ и $BC$ действуют распределённые силы. Тогда граничные условия можно записать в виде
\begin{subequations}
	\renewcommand{\theequation}{\theparentequation\Alph{equation}}
	\begin{equation}\label{l16:eq:15A}
		\hfill\left\{\begin{array}{rcl}
			u(x,0,t)&=&f_1(x,t),\\
			u(x,l_2,t)&=&f_2(x,t),
		\end{array}\right.\hfill
	\end{equation}
	\begin{equation}\label{l16:eq:15B}
		\hfill\left\{\begin{array}{rcl}
			u_x(0,y,t)&=&g_1(y,t),\\
			u_x(l_1,y,t)&=&g_2(y,t).
		\end{array}\right.\hfill
	\end{equation}
\end{subequations}
Начальные условия 
\begin{equation}\label{l16:eq:16}
	\hfill u(P,0)=\phi(P),\quad u_t(P,0)=\psi(P).\hfill
\end{equation}
1.\quad Подбираем функцию $\widetilde{u}(P,t)$ так, чтобы она удовлетворяла граничным условиям либо~\eqref{l16:eq:15A}, либо~\eqref{l16:eq:15B}, то есть на двух противоположных краях мембраны, например на $AD$ и $BC$ то есть~\eqref{l16:eq:15B} и после этого вводим новую неизвестную функцию $v(P,t)$, связанную с исходной функцией $u(P,t)$ равенством 
\begin{equation}\label{l16:eq:17}
	\hfill u=\widetilde{u}+v.\hfill
\end{equation}
Тогда функция $v(P,t)$ будет удовлетворять уравнению
\begin{equation}\label{l16:eq:18}
	\hfill v_{tt}=a^2\cdot\Delta v+\Phi(P,t),\hfill
\end{equation} 
где $\Phi(P,t)=\widetilde{f}(P,t)-\widetilde{u}_{tt}+a^2\cdot\Delta \widetilde{u}$, начальным условиям 
\begin{equation}\label{l16:eq:19}
	\hfill\left.\begin{array}{rcl}
		\underline{v(P,0)=\widetilde{\phi}(P)}&=&\phi(P)-\widetilde{u}(P,0),\\
		\underline{v_t(P,0)=\widetilde{\psi}(P)}&=&\psi(P)-\widetilde{u}_t(P,0)
	\end{array}\right\} \hfill
\end{equation}
и граничным условиям
\begin{equation}\label{l16:eq:20}
	\hfill\begin{array}{rcl}
		\underline{v_x(0,y,t)=0}&=&g_1(y,t)-\widetilde{u}_x(0,y,t),\\
		\underline{v_x(l_1,y,t)=0}&=&g_2(y,t)-\widetilde{u}_x(l_1,y,t);
	\end{array} \hfill
\end{equation}
\begin{equation}\label{l16:eq:21}
	\hfill\begin{array}{rcl}
		\underline{v(x,0,t)=h_1(x,t)}&=&f_1(x,t)-\widetilde{u}(x,0,t),\\
		\underline{v(x,l_2,t)=h_2(x,t)}&=&f_2(x,t)-\widetilde{u}(x,l_2,t).
	\end{array} \hfill
\end{equation}

Нахождение функции $\widetilde{u}$, позволившей свести исходную задачу к задаче~\eqref{l16:eq:18}---\eqref{l16:eq:21} проведём стандартным образом, вспоминая то, как это делалось в случае струны. Ищем $\widetilde{u}$ в виде
\begin{equation*}
	\hfill\widetilde{u}(P,t)=P_1(x)\cdot g_1+P_2(x)\cdot g_2,\hfill
\end{equation*}
где $P_1(x)$ и $P_2(x)$ полиномы, удовлетворяющие требованиям 
\begin{equation*}
	\hfill P'_1(0)=1,\quad P'_2(0)=0,\quad P'_1(l_1)=0,\quad P'_2(l_1)=1\footnotemark{}.\hfill
\end{equation*}\footnotetext{Найдите полиномы $P_j(x)$ самостоятельно.}

Прежде чем решать задачу~\eqref{l16:eq:18}---\eqref{l16:eq:21} сделаем два замечания:
\begin{enumeraterm}
	\item если бы на какой-то паре противоположных краёв мембраны ($AD,\ BC$ или $AB,\ CD$) условия были бы однородными, то есть $f_1=f_2=0$ или $g_1=g_2=0$, то находить функцию $\widetilde{u}$ не надо;
	\item\label{l16:enum:ii} можно было искать функцию $\widetilde{u}$ так, чтобы она удовлетворяла граничным условиям при $y=0$ и $y=l_2$ (то есть на $AB$ и $CD$), тогда функция $v$ (см.~\eqref{l16:eq:17}) удовлетворяла бы нулевым условиям при $y=0$ и $y=l_2$.
\end{enumeraterm}
2.\quad Переходим к решению задачи~\eqref{l16:eq:18}---\eqref{l16:eq:21}. Пусть $X_k(x)$ нормированные собственные функции оператора $\displaystyle-\dder{}{x}$ с граничными условиями порождёнными~\eqref{l16:eq:20}: $X_k'(0)=X_k'(l_1)=0$\footnote{В ситуации замечания~\ref{l16:enum:ii} мы брали бы собственные функции $Y_m(y)$ оператора $-\displaystyle\dder{}{y}$ с граничными условиями $Y_m(0)=Y_m(l_2)=0$.}. Будем искать решение в виде 
\begin{equation}\label{l16:eq:22}
	\hfill v(x,y,t)=\sum\limits_{k=1}^{\infty}v_k(y,t)\cdot X_k(x),\hfill
\end{equation}
то есть в виде разложения искомой функции $v(x,y,t)$ при фиксированных $y,\ t$ по собственным функциям $X_k(x)$ (обоснование --- теорема Стеклова).

Одновременно сделаем аналогичное разложение для функции $\Phi(x,y,t)$ из~\eqref{l16:eq:18}:
\begin{equation}\label{l16:eq:23}
	\hfill\Phi(x,y,t)=\sum\limits_{k=1}^{\infty}\Phi_k(y,t)\cdot X_k(x),\hfill
\end{equation}
где 
\begin{equation*}
	\hfill\Phi_k(y,t)=\int\limits_0^{l_1}\Phi(x,y,t)\cdot X_k(x)\,dx.\hfill
\end{equation*}.
Подставим разложения~\eqref{l16:eq:22},~\eqref{l16:eq:23} в~\eqref{l16:eq:18}. Тогда, считая возможным двукратное почленное дифференцирование этого ряда по $x,\ y,\ t$, получим 
\begin{equation}\label{l16:eq:24}
	\sum\limits_{k=1}^{\infty}\pdder{v_k}{t}\cdot X_k=\sum\limits_{k=1}^{\infty}a^2\cdot\pdder{v_k}{y}\cdot X_k-\sum\limits_{k=1}^{\infty}a^2\cdot\lambda_k\cdot v_k\cdot X_k+\sum\limits_{k=1}^{\infty}\Phi_k(y,t)\cdot X_k,
\end{equation}
где $\lambda_k$ --- собственное значение оператора $-\displaystyle\dder{}{x}$, которому отвечает собственная функция $X_k(x)$. Умножая~\eqref{l16:eq:24} на $X_m(x)$ и интегрируя по $x$ от нуля до $l_1$, получим
\begin{equation}\label{l16:eq:25}
	\hfill\pdder{v_m}{t}=a^2\cdot\pdder{v_m}{y}-a^2\cdot\lambda_m\cdot v_m+\Phi_m.\hfill
\end{equation} 
Найдём граничные и начальные условия для функции $v_m$ ($m$ --- произвольное, но фиксированное), начиная с начальных. В силу~\eqref{l16:eq:19} и~\eqref{l16:eq:22} 
\begin{equation*}
	\hfill\begin{array}{rcccl}
		v(x,y,0)&=&\widetilde{\phi}(x,y)&=&\displaystyle\sum\limits_{k=1}^{\infty}v_k(y,0)\cdot X_k,\\
		v_t(x,y,0)&=&\widetilde{\psi}(x,y)&=&\displaystyle\sum\limits_{k=1}^{\infty}\pder{v_k}{t}(y,0)\cdot X_k.
	\end{array}\hfill
\end{equation*}
Умножая каждое из этих равенств на $X_m$ и интегрируя по $x$ от $0$ до $l_1$ получим 
\begin{equation}\label{l16:eq:26}
	\hfill\left.\begin{array}{rcccl}
		v_m(y,0)&=&\displaystyle\int\limits_0^{l_1}\widetilde{\phi}(x,y)\cdot X_m(x)\,dx& \equiv& \widetilde{\phi}_m(y),\\
		\displaystyle\pder{v_m}{t}(y,0)&=&\displaystyle\int\limits_0^{l_1}\widetilde{\psi}(x,y)\cdot X_m(x)\,dx&\equiv&\widetilde{\psi}_m(y),
	\end{array}\right\}\hfill
\end{equation}
где $\widetilde{\phi}_m(y)$ и $\widetilde{\psi}_m(y)$ --- обозначения для интегралов в~\eqref{l16:eq:26}.

Теперь найдём граничные условия. В силу~\eqref{l16:eq:21},~\eqref{l16:eq:22}
\begin{equation*}
	\hfill\begin{array}{rcccl}
		v(x,0,t)&=&h_1(x,t)&=&\displaystyle\sum\limits_{k=1}^{\infty}v_k(0,t)\cdot X_k,\\
		v(x,l_2,t)&=&h_2(x,t)&=&\displaystyle\sum\limits_{k=1}^{\infty}v_k(l_2,t)\cdot X_k.
	\end{array}\hfill
\end{equation*}
Умножая обе части этих равенств на $X_m$ и интегрируя по $x$ от 0 до $l_1$ получим
\begin{equation}\label{l16:eq:27}
	\hfill\left.\begin{array}{rcccl}
		v_m(0,t)&=&\displaystyle\int\limits_0^{l_1}h_1(x,t)\cdot X_m(x)\,dx& \equiv& \alpha_m(t),\\
		\displaystyle v_m(l_2,t)&=&\displaystyle\int\limits_0^{l_1}h_2(x,t)\cdot X_m(x)\,dx&\equiv&\beta_m(t),
	\end{array}\right\}\hfill
\end{equation}
где $\alpha_m(t)$, $\beta_m(t)$ --- обозначения для интегралов в~\eqref{l16:eq:27}.

Таким образом мы получили для нахождения функции $v_m(y,t)$ следующую задачу: уравнение~\eqref{l16:eq:25} с начальными условиями
\begin{equation*}
	\hfill v_m(y,0)=\widetilde{\phi}_m(y),\quad\pder{v_m}{t}(y,0)=\widetilde{\psi}_m\qquad\text{(это~\eqref{l16:eq:26})}\hfill
\end{equation*}
и граничными условиями 
\begin{equation*}
	\hfill v_m(0,t)=\alpha_m(t),\quad v_m(l_2,t)=\beta_m(t)\qquad\text{(это~\eqref{l16:eq:27})}.\hfill
\end{equation*}

Эта задача очень похожа на задачу о вынужденных колебаниях струны с заданным законом движения концов струны\footnote{Разумеется в предположении, что струна расположена на отрезке $[0,l_2]$ оси $y$.}. Отличие в том, что в уравнении~\eqref{l16:eq:25} присутствует член $-a^2\cdot\lambda_m\cdot v_m$, которого нет в уравнении описывающем вынужденные колебания струны. Однако это ни в малейшей степени не препятствует применению метода Фурье при предварительном переводе неоднородностей из граничных условий в уравнение. Наметим ход решения. Находим функцию $\widetilde{v}_m(y,t)$, удовлетворяющую граничным условиям: 
\begin{equation*}
	\hfill\widetilde{v}_m(0,t)=\alpha_m(t),\quad\widetilde{v}_m(l_2,t)=\beta_m(t).\hfill
\end{equation*}  
Положим
\begin{equation*}
	\hfill \widetilde{v}_m(y,t)=P(y)\cdot\alpha_m(t)+Q(y)\cdot\beta_m(t),\hfill
\end{equation*}
где $P(y)$ и $Q(y)$ --- полиномы, удовлетворяющие условиям 
\begin{equation*}
	\hfill P(0)=1,\quad P(l_2)=0,\quad Q(0)=0,\quad Q(l_2)=1,\hfill
\end{equation*}
где полиномы не зависят от $m$. Далее полагаем 
\begin{equation*}
	\hfill v_m(y,t)=\widetilde{v}_m(y,t)+w_m(y,t),\hfill
\end{equation*}
где $w_m(y,t)$ --- новая неизвестная функция, удовлетворяющая по построению нулевым граничным условиям
\begin{equation}\label{l16:eq:28}
	\hfill w_m(0,t)=w_m(l_2,t)=0,\hfill
\end{equation}
начальным условиям 
\begin{equation}\label{l16:eq:29}
	\hfill\left.\begin{array}{rcccl}
		w_m(y,0)&=&\widetilde{\phi}_m(y)-\widetilde{v}_m(y,0)&\equiv&\widehat{\phi}(y),\\
		\displaystyle \pder{w_m}{t}(y,0)&=&\displaystyle\widetilde{\psi}_m(y)-\pder{\widetilde{v}_m}{t}(y,0)&\equiv&\psihat_m(y)
	\end{array}\right\}\hfill
\end{equation}
и уравнению (в силу~\eqref{l16:eq:25})
\begin{equation}\label{l16:eq:30}
	\hfill\pdder{w_m}{t}=a^2\cdot\pdder{w_m}{y}-a^2\cdot\lambda_m\cdot w_m+\widehat{\Phi}_m,\hfill
\end{equation}
где 
\begin{equation*}
	\hfill\widehat{\Phi}_m(y,t)=\Phi_m(y,t)-\pdder{\widetilde{v}_m}{t}+a^2\cdot\pdder{\widetilde{v}_m}{y}-a^2\cdot\lambda_m\cdot \widetilde{v}_m.\hfill
\end{equation*}

Дальнейший ход решения не отличается от известного для струны. То есть ищем $w_m(y,t)$ в виде ряда по собственным функциям оператора $\displaystyle-\dder{}{y}$ с условиями $Y_k(0)=Y_k(l_2)=0$:
\begin{equation*}
	\hfill w_m(y,t)=\sum\limits_{k=1}^{\infty}b_{mk}\cdot Y_k(y)\hfill
\end{equation*}
и так далее\dots
\vspace{0.2cm}

\noindent\textbf{Задание.} Довести решение до конца\footnote{На экзамене почти никто не мог решить подобную задачу\dots}.