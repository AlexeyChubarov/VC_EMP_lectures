	\chapter{}
\label{lecture13}
\section{Свободные колебания неоднородной струны.}
\label{lecture13section1}
На прошлой лекции мы изучали свободные колебания \emph{однородной} струны. А если она не однородна? В этом случае натяжение $\mu$ и плотность $\rho$ не являются константами. Для простоты будем предполагать, что $\mu=\mu(x)$, $\rho=\rho(x)$\footnote[1]{То есть нет зависимости от времени.}. Тогда уравнение свободных колебаний струны запишется в виде
\begin{equation}\label{l13:eq:1}
	\hfill\rho\cdot\pdder{u}{t}=\pder{}{x}\big(\mu\cdot u_x\big).\hfill
\end{equation}
Будем решать это уравнение при начальных условиях
\begin{equation}\label{l13:eq:2}
	\hfill u(x,0)=\phi(x),\quad u_t(x,0)=\psi(x)\hfill
\end{equation}
и граничных условиях
\begin{equation}\label{l13:eq:3}
	\hfill u(0,t)=0,\quad u(l,t)=0\footnotemark{}.\hfill
\end{equation}\footnotetext{Мы как и ранее считаем, что струна в невозмущенном состоянии занимает отрезок $[0,l]$ оси $x$.}

Для решения задачи~\eqref{l13:eq:1}---\eqref{l13:eq:3} применим метод Фурье. Как и раньше, отыскиваем сначала частное решение $u_{\text{ч}}(x,t)$, которое удовлетворяет~\eqref{l13:eq:1},~\eqref{l13:eq:3}. Пусть $u_{\text{ч}}=X(x)\cdot T(t)$. Подставляя в~\eqref{l13:eq:1} и деля потом обе части~\eqref{l13:eq:1}на произведение $\rho\cdot X\cdot T$, получим
\begin{equation}\label{l13:eq:4}
	\hfill\frac{T''}{T}=\frac{\displaystyle\pder{}{x}\big(\mu\cdot X'\big)}{\rho\cdot X}.\hfill
\end{equation} 
Как и в случае однородной струны, убеждаемся, что эти отношения не зависят ни от $x$, ни от $t$, то есть равны константе, которую мы обозначим через $-\lambda$. Тогда в силу~\eqref{l13:eq:4} 
\begin{equation}\label{l13:eq:5}
	\hfill T''+\lambda\cdot T=0,\hfill
\end{equation}
\begin{equation}\label{l13:eq:6}
	\hfill -\der{}{x}\big(\mu\cdot X'\big)=\lambda\cdot\rho\cdot X.\hfill
\end{equation}
Из условия~\eqref{l13:eq:3} получаем
\begin{equation}\label{l13:eq:7}
	\hfill X(0)=0,\quad X(l)=0.\hfill
\end{equation}
Таким образом видим, что $\lambda$ и $X(x)$ есть соответственно собственное значение и собственная функция обобщённой задачи Штурма для оператора $\displaystyle-\der{}{x}\big(\mu\cdot X'\big)$. А мы знаем, что для обобщённой задачи Штурма существует бесконечная серия собственных значений $\lambda_k$, $k=1,2,\ldots$ и соответствующих собственных функций $X_k(x)$, причём при $\lambda_k\neq\lambda_m$ функции $X_k,\ X_m$ будут ортогональны с весом $\rho$:
\begin{equation*}
	\hfill\int\limits_0^l X_k(x)\cdot X_m(x)\cdot\rho(x)\,dx=0,\quad k\neq m.\hfill
\end{equation*}
Подставив $\lambda_k$ в~\eqref{l13:eq:5} получим
\begin{equation*}
	\hfill T_k''+\lambda_k\cdot T_k=0,\hfill
\end{equation*}
откуда 
\begin{equation*}
	\hfill T_k(t)=A_k\cdot\cos\left(\sqrt{\lambda_k}\cdot t\right)+B_k\cdot\sin\left(\sqrt{\lambda_k}\cdot t\right).\hfill
\end{equation*}
Теперь 
\begin{equation*}
	\hfill u_{\text{ч}}(x,t)=u_k(x,t)=X_k(x)\cdot T_k(t).\hfill
\end{equation*}
Ищем решение задачи~\eqref{l13:eq:1}---\eqref{l13:eq:3} в виде ряда
\begin{equation*}
	\hfill u(x,t)=\sum\limits_{k=1}^{\infty}X_k(x)\cdot T_k(t).\hfill
\end{equation*}
Если коэффициенты $A_k$, $B_k$ обеспечивают равномерную сходимость этого ряда и возможность его двукратного почленного дифференцирования по $x$ и по $t$, то функция $u(x,t)$ есть решение задачи~\eqref{l13:eq:1},~\eqref{l13:eq:3}. Чтобы обеспечить выполнение условий~\eqref{l13:eq:2} найдём коэффициенты $A_k$ и $B_k$ из соотношений 
\begin{equation*}
	\hfill u(x,0)=\sum\limits_{k=1}^{\infty}A_k\cdot X_k(x)=\phi(x),\quad u_t(x,0)=\sum\limits_{k=1}^{\infty}\sqrt{\lambda_k}\cdot B_k\cdot X_k(x)=\psi(x). \hfill
\end{equation*}
Эти соотношения суть разложения функций $\phi(x)$ и $\psi(x)$ по собственным функциям обобщённой задачи Штурма. В силу теоремы Стеклова такие разложения имеют место и коэффициенты $A_k$ и $B_k$ определяются равенствами
\begin{equation*}
	\hfill A_k=\frac{\displaystyle\smallint\limits_0^l\phi(s)\cdot X_k(s)\cdot\rho(s)\,ds}{\displaystyle\smallint\limits_0^l X^2_k(s)\cdot\rho(s)\,ds},\quad B_k=\frac{\displaystyle\smallint\limits_0^l\psi(s)\cdot X_k(s)\cdot\rho(s)\,ds}{\displaystyle\sqrt{\lambda_k}\cdot\smallint\limits_0^l X^2_k(s)\cdot\rho(s)\,ds}.\hfill
\end{equation*}
\section{Вынужденные колебания однородной струны.}
\label{lecture13section2}
Как мы уже говорили на прошлой лекции, если на однородную струну действует распределённая сила с линейной плотностью $\widetilde{f}(x,t)$, то в уравнение колебаний струны войдёт плотность $f(x,t)=\widetilde{f}\!\bigm/\!\rho$ на единицу массы. Таким образом, мы будем решать уравнение 
\begin{equation}\label{l13:eq:8}
	\hfill\pdder{u}{t}=a^2\cdot\pdder{u}{x}+f(x,t).\hfill
\end{equation}
Начальные условия:
\begin{equation}\label{l13:eq:9}
	\hfill u(x,0)=\phi(x),\quad u_t(x,0)=\psi(x).\hfill
\end{equation}
Что касается граничных условий, то мы будем рассматривать или задание закона движения концов струны
\begin{equation}\label{l13:eq:10}
	\hfill u(0,t)=\nu_1(t),\quad u(l,t)=\nu_2(t)\hfill
\end{equation}
или наличие на концах действующих сосредоточенных сил
\begin{equation}\label{l13:eq:10a}
	\hfill u_x(0,t)=-f_1(t)\!\bigm/\!\mu,\quad u_x(l,t)=f_2(t)\!\bigm/\!\mu\footnotemark{}.\hfill\tag{\theequation a}
\end{equation}\footnotetext{Заметим, кстати, что студенты часто делают ошибку, забывая при записи граничных условий~\eqref{l13:eq:10a} поделить правую часть на $\mu$.} Для краткости положим $\widetilde{f}_1\eqdef -f_1\!\bigm/\!\mu$, $\widetilde{f}_2\eqdef f_2\!\bigm/\!\mu$ и перепишем~\eqref{l13:eq:10a} в виде
\begin{equation}\label{l13:eq:11}
	\hfill u_x(0,t)=\widetilde{f}_1(t),\quad u_x(l,t)=\widetilde{f}_2(t)\hfill
\end{equation}
При решении задачи~\eqref{l13:eq:8}---\eqref{l13:eq:10}, или~\eqref{l13:eq:8},~\eqref{l13:eq:9},~\eqref{l13:eq:11} в первую очередь надо перевести неоднородности из граничных условий в уравнение. Для этого мы отыскиваем функцию $v_0(x,t)$, удовлетворяющую граничным условиям~\eqref{l13:eq:10} или~\eqref{l13:eq:11}  и затем вводим новую неизвестную функцию $v(x,t)$ соотношением 
\begin{equation*}
	\hfill v=u-v_0.\hfill
\end{equation*}  
Тогда $u=v+v_0$ и подставив это выражение в~\eqref{l13:eq:8} и~\eqref{l13:eq:9} мы получим для функции $v$ уравнение 
\begin{equation}\label{l13:eq:12}
	\hfill\pdder{v}{t}=a^2\cdot\pdder{v}{x}+g(x,t)\hfill
\end{equation}
и начальные условия
\begin{equation}\label{l13:eq:13}
	\hfill v(x,0)=\widetilde{\phi},\quad v_t(x,0)=\widetilde{\psi},\hfill
\end{equation}
где $\displaystyle g(x,t)=f-\pdder{v_0}{t}+a^2\cdot\pdder{v_0}{x}$, $\widetilde{\phi}=\phi(x)-v_0(x,0)$, $\displaystyle\widetilde{\psi}=\psi(x)-\pder{v_0}{t}(x,0)$. Но главным является тот факт, что для функции $v$ граничные условия --- однородны. Действительно, подставляя $u=v+v_0$ в~\eqref{l13:eq:10} и в~\eqref{l13:eq:11} имеем, если $v_0(x,t)$ удовлетворяет~\eqref{l13:eq:10}\addtocounter{equation}{-3}
\begin{equation}\label{l13:eq:10A}
	\hfill v(0,t)+v_0(0,t)=\nu_1(t),\quad v(l,t)+v_0(l,t)=\nu_2(t);\hfill\tag{\theequation A}
\end{equation}\addtocounter{equation}{1}
или если $v_0(x,t)$ удовлетворяет~\eqref{l13:eq:11}
\begin{equation}\label{l13:eq:11A}
	\hfill v_x(0,t)+v_{0x}(0,t)=\widetilde{f}_1(t),\quad v_x(l,t)+v_{0x}(l,t)=\widetilde{f}_2(t).\hfill\tag{\theequation A}
\end{equation}\addtocounter{equation}{2}
Но поскольку функция $v_0$ удовлетворяет граничным условиям задачи, то если мы решали задачу с условием~\eqref{l13:eq:10}, тогда
\begin{equation*}
	\hfill v_0(0,t)=\nu_1(t),\quad v_0(l,t)=\nu_2(t)\hfill
\end{equation*} 
и поэтому из~\eqref{l13:eq:10A} следует, что
\begin{equation}\label{l13:eq:14}
	\hfill v(0,t)=0,\quad v(l,t)=0.\hfill
\end{equation}
Аналогично, если мы решали задачу с условиями~\eqref{l13:eq:11}, то 
\begin{equation*}
	\hfill v_{0x}(0,t)=\widetilde{f}_1(t),\quad v_{0x}(l,t)=\widetilde{f}_2(t)\hfill
\end{equation*} 
и поэтому вследствие~\eqref{l13:eq:11A} 
\begin{equation}\label{l13:eq:15}
	\hfill v_x(0,t)=0,\quad v_x(l,t)=0.\hfill
\end{equation}
Таким образом для функции $v(x,t)$ мы получили задачу с однородными граничными условиями. Прежде чем описывать метод решения задачи~\eqref{l13:eq:12}---\eqref{l13:eq:14} (или~\eqref{l13:eq:12},~\eqref{l13:eq:13},~\eqref{l13:eq:15}) дадим рецепт построения функции $v_0(x,t)$ на примере граничных условий~\eqref{l13:eq:10} (случай условий~\eqref{l13:eq:11} не имеет принципиальных отличий). Функцию $v_0(x,t)$ будем искать в виде
\begin{equation*}
	\hfill v_0(x,t)=P_1(x)\cdot\nu_1(t)+P_2(x)\cdot\nu_2(t)\text{ --- в случае~\eqref{l13:eq:10}}\hfill
\end{equation*}
\begin{equation*}
	\hfill \left(\text{в случае~\eqref{l13:eq:11}: }v_0(x,t)=P_3(x)\cdot\widetilde{f}_1(t)+P_4(x)\cdot\widetilde{f}_2(t)\right),\hfill
\end{equation*}
где $P_i(x)$ --- полиномы (как правило первой степени). Требуем
\begin{equation*}
	\hfill \begin{array}{rcccl}
		v_0(0,t)&=&\nu_1(t)&=&P_1(0)\cdot\nu_1(t)+P_2(0)\cdot\nu_2(t),\\	v_0(l,t)&=&\nu_2(t)&=&P_1(l)\cdot\nu_1(t)+P_2(l)\cdot\nu_2(t).
	\end{array}\hfill
\end{equation*}
Отсюда ясно, что достаточно выбрать полиномы $P_i(x)$ так, чтобы 
\begin{equation*}
	\hfill P_1(0)=1,\quad P_2(0)=0,\quad P_1(l)=0,\quad P_2(l)=1.\hfill
\end{equation*}
Взяв $P_i(x)=\alpha_i\cdot x+\beta_i$, $i=1,2$ мы получаем
\begin{equation*}
	\hfill P_1(x)=-\frac{x}{l}+1,\quad P_2(x)=\frac{x}{l}.\hfill
\end{equation*}
Таким образом функция $v_0(x,t)$, удовлетворяющая условиям~\eqref{l13:eq:10} построена.
\vspace{0.2cm}

\noindent\textbf{Задание.} Найти функцию $v_0(x,t)$, удовлетворяющую условиям~\eqref{l13:eq:11}.
\vspace{0.2cm}

Возвращаемся теперь к задаче~\eqref{l13:eq:12}---\eqref{l13:eq:14}\footnote{Задача~\eqref{l13:eq:12},~\eqref{l13:eq:13},~\eqref{l13:eq:15} решается аналогично. О некоторых отличиях скажем позднее.}. Самое главное здесь --- решить задачу~\eqref{l13:eq:12},~\eqref{l13:eq:14} (или~\eqref{l13:eq:12},~\eqref{l13:eq:15}), то есть найти частное решение, удовлетворяющее уравнению и граничным условиям задачи, не заботясь о выполнении начальных условий~\eqref{l13:eq:13} или выбирая начальные условия любым образом. Действительно, пусть функция $v_1(x,t)$ удовлетворяет~\eqref{l13:eq:12},~\eqref{l13:eq:14} (или~\eqref{l13:eq:12},~\eqref{l13:eq:15}). Будем искать решение нашей задачи в виде
\begin{equation*}
	\hfill v(x,t)=v_{2}(x,t)+v_1(x,t),\hfill
\end{equation*}
где $v_2(x,t)$ --- новая неизвестная функция. Подставляя в~\eqref{l13:eq:12} выражение для $v(x,t)$ получаем:
\begin{equation*}
	\hfill\pdder{v_2}{t}+\pdder{v_1}{t}=a^2\cdot\pdder{v_2}{x}+a^2\cdot\pdder{v_1}{x}+g(x,t),\hfill
\end{equation*}
но
\begin{equation*}
	\hfill\pdder{v_1}{t}=a^2\cdot\pdder{v_1}{x}+g(x,t),\hfill
\end{equation*}
поэтому на <<долю>> $v_2$ остаётся уравнение
\begin{equation}\label{l13:eq:16}
	\hfill\pdder{v_2}{t}=a^2\cdot\pdder{v_2}{x}.\hfill
\end{equation}
Что касается граничных условий, то имеем в силу~\eqref{l13:eq:14} (или~\eqref{l13:eq:15})
\begin{gather*}
	 v_2(0,t)+v_1(0,t)=0,\quad v_2(l,t)+v_1(l,t)=0\hfill
\intertext{или} 
 \pder{v_2}{x}(0,t)+\pder{v_1}{x}(0,t)=0,\quad \pder{v_2}{x}(l,t)+\pder{v_1}{x}(l,t)=0.\hfill
\end{gather*} 
Но так как функция $v_1$ удовлетворяет граничным условиям, то для $v_2(x,t)$ мы получим 
\begin{equation}\label{l13:eq:17}
	\hfill v_2(0,t)=v_2(l,t)=0\hfill
\end{equation}
или
\begin{equation}\label{l13:eq:18}
	\hfill \pder{v_2}{x}(0,t)=\pder{v_2}{x}(l,t)=0.\hfill
\end{equation}
Наконец из начальных условий~\eqref{l13:eq:13} для функции $v(x,t)$ 
\begin{equation*}
	\hfill v_2(x,0)=\widetilde{\phi}(x)-v_1(x,0),\quad \pder{v_2}{t}(x,0)=\widetilde{\psi}(x)-\pder{v_1}{t}(x,0).\hfill
\end{equation*}
Таким образом для функции $v_2(x,t)$ мы получили задачу, решению которой была посвящена прошлая лекция, и мы можем написать
\begin{equation*}
	\hfill v_2(x,t)=\int\limits_0^l G(x,\xi,t)\cdot\psihat(\xi)\,d\xi+\int\limits_0^l G_t(x,\xi,t)\cdot\widehat{\phi}(\xi)\,d\xi,\hfill
\end{equation*}
где $\displaystyle \psihat=\widetilde{\psi}-\pder{v_1}{t}(x,0)$, $\displaystyle\widehat{\phi}=\widetilde{\phi}-v_1(x,0)$, а функция Грина $G(x,\xi,t)$ построена для граничных условий~\eqref{l13:eq:17} или~\eqref{l13:eq:18}. 

Заметим, что если нам удаётся найти $v_1(x,t)$ так, чтобы выполнялись условия~\eqref{l13:eq:13}, то $v_1(x,t)$ и есть решение поставленной задачи, а $v_2(x,t)\equiv0$.

Теперь, наконец, о способах решения задачи~\eqref{l13:eq:12},~\eqref{l13:eq:14} (или~\eqref{l13:eq:12},~\eqref{l13:eq:15}).

Рассмотрим сначала частный, но важный случай, когда в~\eqref{l13:eq:12} $g(x,t)=g(x)$. В этом случае попробуем найти стационарное решение $v(x,t)=v_1(x)$. Тогда 
\begin{equation*}
	\hfill a^2\dder{v_1}{x}+g(x)=0\quad\text{и }v_1(0)=v_1(l)=0\ \left(\text{или }\pder{v_1}{x}(0)=\pder{v_1}{x}(l)=0\right).\hfill
\end{equation*} 
Очевидно
\begin{equation*}
	\hfill v_1(x)=-\int\limits_0^x ds\int\limits_0^s\frac{g(\alpha)}{a^2}\,d\alpha+C_1\cdot x+C_2.\hfill
\end{equation*}
При закреплённых концах
\begin{equation*}
	\hfill C_2=0,\quad C_1=\frac{1}{a^2}\int\limits_0^l ds\int\limits_0^s g(\alpha)\,d\alpha\cdot\frac{1}{l}.\hfill
\end{equation*}
Однако если концы свободны (оба!), то найти стационарное решение как правило не удаётся. Действительно,
\begin{equation*}
	\hfill\pder{v_1}{x}=-\int\limits_0^x g(\alpha)\,d\alpha\cdot\frac{1}{a^2}+C_1.\hfill
\end{equation*}
При $x=0$ получаем $C_1=0$, а при $x=l$
\begin{equation}\label{l13:eq:ast}
	\hfill\int\limits_0^l g(\alpha)\,d\alpha=0.\tag{$\ast$}\hfill
\end{equation}
Если равенство~\eqref{l13:eq:ast} выполняется,то стационарное решение существует, если нет --- не существует. Условие~\eqref{l13:eq:ast} имеет простой физический смысл --- это равенство нулю равнодействующей силы, действующей на струну. Ясно, что если эта равнодействующая сила не нулевая, то при свободных концах струна начнёт двигаться как целое и стационарного решения быть не может.
\vspace{0.2cm}

\noindent\textbf{Задание.} Проверить, существует ли стационарное решение при $g(x,t)=g(x)$, если закреплён один из концов струны.
\vspace{0.2cm}

Переходим к общему методу решения задачи~\eqref{l13:eq:12},~\eqref{l13:eq:14} (или~\eqref{l13:eq:12},~\eqref{l13:eq:15}). В первую очередь надо найти собственные значения и собственные функции оператора $\displaystyle-\dder{}{x}$ с граничными условиями задачи (то есть~\eqref{l13:eq:14} или~\eqref{l13:eq:15}). Пусть эти собственные значения $\lambda_k$ и нормированные собственные функции $X_k(x)$\footnote{В условиях~\eqref{l13:eq:14} $\displaystyle X_k(x)=\sqrt{\frac{2}{l}}\cdot\sin\left(\frac{\pi\cdot k}{l}\cdot x\right)$, $\displaystyle\lambda_k=\left(\frac{\pi\cdot k}{l}\cdot x\right)^2$.}. Будем искать функцию $v_1$ в виде ряда
\begin{equation*}
	\hfill v_1(x,t)=\sum\limits_{k=1}^{\infty}b_k(t)\cdot X_k(x).\hfill
\end{equation*}
При каждом фиксированном значении $t$ этот ряд есть разложение (по теореме Стеклова) функции $v_1(x,t)$, как функции от $x$ по собственным функциям оператора Штурма, где <<обобщённые коэффициенты Фурье>> --- $b_k(t)$ --- естественно зависят от $t$. Наша цель --- найти $b_k(t)$. Далее разложим функцию $g(x,t)$ в~\eqref{l13:eq:12} аналогичным образом
\begin{equation*}
	\hfill g(x,t)=\sum\limits_{k=1}^{\infty}g_k(t)\cdot X_k(x),\hfill
\end{equation*}
где $\displaystyle g_k(t)=\smallint\limits_0^l g(\xi,t)\cdot X_k(\xi)\,d\xi$ --- известные коэффициенты.

Далее подставим разложения $v_1(x,t)$ и $g(x,t)$ в уравнение~\eqref{l13:eq:12} и считаем, что ряд для $v_1(x,t)$ допускает двукратное почленное дифференцирование. Получим, учитывая, что $X_k''=-\lambda_k\cdot X_k$
\begin{equation*}
	\hfill\sum\limits_{k=1}^{\infty}b_k''(t)\cdot X_k(x)=-a^2\cdot\sum\limits_{k=1}^{\infty} \lambda_k\cdot b_k(t)\cdot X_k(x)+\sum\limits_{k=1}^{\infty}g_k(t)\cdot X_k(x).\hfill
\end{equation*}
Умножая это соотношение на $X_n(x)$ скалярно, мы получим уравнение на $b_n(t)$:
\begin{equation*}
	\hfill b_n''(t)=-a^2\cdot\lambda_k\cdot b_n(t)+g_n(t).\hfill
\end{equation*}
или, обозначая $a^2\cdot\lambda_n$ через $\omega_n^2$,
\begin{equation}\label{l13:eq:19}
	\hfill b_n''(t)+\omega_n^2\cdot b_n(t)=g_n(t).\hfill
\end{equation}
Мы получили для неизвестной функции $b_n(t)$ дифференциальное уравнение второго порядка, для нахождения единственного решения которого нужны начальные условия. 

Из разложения для функций $v_1(x,t)$ следует, что
\begin{equation}\label{l13:eq:20}
	\hfill b_n(0)=\int\limits_0^l v_1(x,0)\cdot X_n(x)\,dx,\quad b'_n(0)=\int\limits_0^l \pder{v_1}{t}(x,0)\cdot X_n(x)\,dx.\hfill
\end{equation}
Далее есть два пути решения.
\begin{enumerateD}
	\item Если взять для $v_1(x,t)$ в качестве начальных условий условия~\eqref{l13:eq:13}
	\begin{equation}\label{l13:eq:21}
		\hfill v_1(x,0)=\widetilde{\phi}(x),\quad\pder{v_1}{t}(x,0)=\widetilde{\psi}(x),\hfill
	\end{equation}
	то после нахождения решения~\eqref{l13:eq:19} с начальными условиями~\eqref{l13:eq:20} с учётом~\eqref{l13:eq:21} мы получим функции $b_n(t)$, которые после подстановки в ряд для $v_1(x,t)$ дадут функцию $v_1(x,t)$, являющуюся решением~\eqref{l13:eq:12}---\eqref{l13:eq:14} или~\eqref{l13:eq:12},~\eqref{l13:eq:13},~\eqref{l13:eq:15}. В этом случае в равенстве 
	\begin{equation*}
		\hfill v(x,t)=v_1(x,t)+v_2(x,t)\hfill
	\end{equation*} 
	составляющей $v_2(x,t)$ не будет ($v_2(x,t)\equiv0$).
	\item Если взять для $v_1(x,t)$ нулевые начальные условия, то это упрощает решение~\eqref{l13:eq:19}, но тогда к полученному в итоге ряду $v_1(x,t)$ надо будет добавить решение $v_2(x,t)$ однородного уравнения с начальными и граничными условиями задачи.   
\end{enumerateD}

\noindent Доведём до конца решение выбрав второй путь.  

Итак берём $\displaystyle v_1(x,0)=\pder{v_1}{t}(x,0)=0$. Тогда из~\eqref{l13:eq:20} получаем, что $b_n(0)=0$, $b_n'(0)=0$ и решение~\eqref{l13:eq:19} с этими начальными условиями запишется в виде 
\begin{equation*}
	b_n(t)=\frac{1}{\omega_n}\cdot\int\limits_0^t\sin\left(\omega_n\cdot(t-\tau)\right)\cdot g_n(\tau)\,d\tau.
\end{equation*}
Подставляя это выражение в ряд для $v_1(x,t)$ и одновременно заменяя здесь $g_n(\tau)$ его выражением получим 
\begin{equation}\label{l13:eq:22}
	\hfill v_1(x,t)=\sum\limits_{k=1}^{\infty}\int\limits_{0}^t\!\int\limits_0^{l}\frac{1}{\omega_k}\cdot\sin\left(\omega_k\cdot(t-\tau)\right)\cdot X_k(\xi)\cdot X_k(x)\cdot g(\xi,\tau)\,d\xi d\tau.\hfill
\end{equation}
Для того, чтобы получить решение $v=v_2+v_1$ исходной задачи вспомним, что $v_2(x,t)$ как решение однородного уравнения выражалось равенством
\begin{equation}\label{l13:eq:23}
	\hfill v_2(x,t)=\int\limits_0^l G(x,\xi,t)\cdot\psi(\xi)\,d\xi+\int\limits_0^l\pder{G}{t}(x,\xi,t)\cdot\phi(\xi)\,d\xi,\hfill
\end{equation}
где 
\begin{equation}\label{l13:eq:24}
	\hfill G(x,\xi,t)=\sum\limits_{k=1}^{\infty}\frac{1}{\omega_k}\cdot\sin\left(\omega_k\cdot t\right)\cdot X_k(\xi)\cdot X_k(x),\quad\omega_k=a\cdot\sqrt{\lambda_k}.\hfill
\end{equation}
Если в~\eqref{l13:eq:22} можно внести суммирование под знак двойного интеграла, то~\eqref{l13:eq:22} можно записать в виде
\begin{equation}\label{l13:eq:25}
	\hfill v_1(x,t)=\int\limits_0^t\!\int\limits_0^l G(x,\xi,t-\tau)\cdot g(\xi,\tau)\,d\xi d\tau.\hfill
\end{equation} 
Таким образом окончательно можно записать, что решение исходной задачи $v(x,t)$ есть сумма $v_1(x,t)$ и $v_2(x,t)$, где $v_1(x,t)$ даётся равенством~\eqref{l13:eq:25}, а $v_2(x,t)$ --- равенством~\eqref{l13:eq:23}\footnote{Напоминаю: интегралы от $G(x,\xi,t)\cdot\psi(\xi)$, $G_t(x,\xi,t)\cdot\phi(\xi)$, $G(x,\xi,t-\tau)\cdot g(\xi,\tau)$ --- это пределы при $n\to\infty$ интегралов от сумм $S_n(x,\xi,t)\cdot\psi(\xi)$, $\displaystyle\pder{S_n}{t}(x,\xi,t)\cdot\phi(\xi)$, $S_n(x,\xi,t-\tau)\cdot g(\xi,\tau)$, где
	\begin{equation*}
		\hfill S_n(x,\xi,t)=\sum\limits_{k=1}^n\frac{1}{\omega_k}\cdot\sin\left(\omega_k\cdot t\right)\cdot X_k(x)\cdot X_k(\xi)\quad\text{(см.~\eqref{l13:eq:24})}.\hfill
\end{equation*}}. Мы видим, что главное --- найти собственные значения и собственные функции оператора $-\displaystyle\dder{}{x}$ с граничными условиями задачи, ибо после этого можно построить функцию Грина. Мы не останавливаемся здесь на физическом смысле функции Грина, но позже к этому вопросу вернёмся.

Заканчивая изучение колебаний струны заметим, что вывод уравнения и граничных условий мы приводили с помощью принципа Гамильтона, который применим только когда силы --- потенциальны. Поэтому, например, уравнение колебаний струны в среде с сопротивлением пропорциональным скорости нельзя получить с помощью принципа Гамильтона и я привожу его без вывода
\begin{equation*}
	\hfill\pdder{u}{t}=a^2\cdot\pdder{u}{x}-q\cdot\pder{u}{t}+f(x,t),\hfill
\end{equation*} 
где $q>0$ --- некоторая константа. Решение этого уравнения может быть проведено методом Фурье и я советую это проделать.