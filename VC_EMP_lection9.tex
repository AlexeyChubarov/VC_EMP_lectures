\documentclass[12pt,a4paper,openany,fleqn]{book}

\usepackage{cmap}
\usepackage[left=1.3cm, right=1.3cm, top=1.4cm, bottom=1.3cm]{geometry}
\usepackage[utf8]{inputenc}
\usepackage[english, russian]{babel}
\usepackage{indentfirst}
\usepackage{amsmath}
\usepackage{amssymb}
\usepackage{amsthm}
\usepackage{gensymb}
\usepackage{mathrsfs}
\usepackage{bm}
\usepackage{perpage}
\usepackage{enumitem}
\usepackage[unicode, colorlinks=true]{hyperref}
\usepackage{tikz}
\usepackage{tabularx}
\usepackage{graphicx}
\usepackage{xfrac}


\setcounter{tocdepth}{1}

\setlength{\mathindent}{0em}

\addto{\captionsrussian}{\renewcommand{\contentsname}{Содержание}}
\addto{\captionsrussian}{\renewcommand{\chaptername}{Лекция}}

\newcommand {\defeq}{\stackrel{\hspace{0.09 cm}de\!f}{=}}
\newcommand {\eqdef}{\defeq}
\newcommand{\iffdef}{\stackrel{\hspace{0.09 cm}de\!f}{\iff}}
\newcommand{\R}{\ensuremath{\mathbb{R}}}
\newcommand{\Cf}{\ensuremath{\mathcal{C}}}
\newcommand{\J}{\ensuremath{\mathcal{J}}}
\newcommand{\mc}[1]{\ensuremath{\mathcal{#1}}}
\newcommand{\Cfn}[2][]{\ensuremath{\Cf{\mathstrut}^{#2}_{#1}}}
\newcommand{\der}[2]{\ensuremath{\frac{d#1}{d#2}}}
\newcommand{\dder}[2]{\ensuremath{\frac{d^2#1}{d#2^2}}}
\newcommand{\pder}[2]{\ensuremath{\frac{\partial#1}{\partial#2}}}
\newcommand{\pdder}[2]{\ensuremath{\frac{\partial^2#1}{\partial#2^2}}}
\newcommand{\eps}{\varepsilon}
\newcommand{\K}{\mc{K}}
\newcommand{\LL}{\ensuremath{L}}
\newcommand{\fL}[1][{[a,b]}]{\ensuremath{\mathscr{L}\hspace*{-0.11 cm}{\mathstrut}_{2}{\scriptstyle#1}}}
\newcommand{\norm}[1]{\ensuremath{\left\|#1\right\|}}
\newcommand{\Ul}[1][\lambda]{\ensuremath{\mc{U}\!{\mathstrut}_{#1}}}
\newcommand{\fLr}[1][{[a,b];\rho}]{\ensuremath{\mathscr{L}\hspace*{-0.11 cm}{\mathstrut}_{2}{\scriptstyle\left(\vphantom{A^b}#1\right)}}}

\DeclareMathOperator\Tg{tg}
\DeclareMathOperator{\Div}{div}


\pagestyle{headings}

\theoremstyle{definition}
\newtheorem{_def}{Определение}[section]
\newtheorem{_lemm}{Лемма}[section]
\newtheorem{_teor}{Теорема}[section]
\newtheorem*{_rem}{Замечание}
\newtheorem*{_con}{Следствие}
\MakePerPage{footnote}

\setlist[enumerate,1]{label=\alph*), ref=\alph*)}

\newlist{enumerate1}{enumerate}{1}
\setlist[enumerate1,1]{label=\arabic*), ref=\arabic*)}
\newlist{enumerateA}{enumerate}{1}
\setlist[enumerateA,1]{label=\Alph*), ref=\Alph*)}

\newlist{enumerateD}{enumerate}{2}
\setlist[enumerateD,1]{label=\arabic*., ref=\arabic*.}
\setlist[enumerateD,2]{{label=\alph*., ref=\alph*.}}

\newlist{enumerateBr}{enumerate}{1}
\setlist[enumerateBr,1]{label=\underline{(\arabic*)}, ref=\underline{(\arabic*)}}

\newlist{enumerateP1}{enumerate}{1}
\setlist[enumerateP1,1]{label=п.\,\arabic*.,ref=п.\,\arabic*.}

\newlist{enumeraterm}{enumerate}{1}
\setlist[enumeraterm,1]{label=\roman*),ref=\roman*)}
\begin{document}
	\author{Г.\,М.~Жислин}
	\title{Лекции по вариационному исчислению и уравнениям математической физики}
	\date{Конспектировал А.\,Г.~Чубаров}
	
	
	
	\maketitle
	
	
	\renewcommand{\thepart}{\Asbuk{part}}
	\renewcommand{\thechapter}{\arabic{chapter}}
	\renewcommand{\thesection}{\arabic{section}}
	\renewcommand{\thesubsection}{\Roman{subsection}}
	\renewcommand{\thefootnote}{\roman{footnote}}
	\renewcommand{\phi}{\varphi}
	\renewcommand{\Re}{\ensuremath{\mc{R}e\,}}
	\renewcommand{\Im}{\ensuremath{\mc{I}m\,}}
	\numberwithin{equation}{section}
	
	\setcounter{chapter}{8}
	\chapter{}
	\label{lecture9}
	\section[Функционал Бесселя. Уравнение Бесселя. (Продолжение.)]{Квадратичный функционал специального вида. Уравнение Бесселя. (Продолжение.)}
	\label{lecture9section1}
	Итак рассматривая задачу на
	\begin{multline*}
		\min\limits_{y\in\K}\J[y],\quad\text{для}\quad\J[y]=\int\limits_0^R\left(x\cdot y^{\prime2}+\frac{\nu^2}{x}\cdot y^2\right)\,dx,\\
		\K=\left\{y(x)|y(x)\in\Cfn[{[0,R]}]{},\ y(R)=0,\ \begin{array}{rcl}
			\nu>0&:&y(0)=0,\ y\in\Cfn[{(0,R]}]{1},\\
			\nu=0&:&y\in\Cfn[{[0,R]}]{1},
		\end{array}\ \int\limits_0^R x\cdot y^2\,dx=1\right\}_{\displaystyle.}
	\end{multline*} 
	Мы установили, что минимайзер должен удовлетворять уравнению
	\begin{equation}\label{l9:eq:1}
		\hfill Ly\eqdef-\der{}{x}\Big(x\cdot y'\Big)+\frac{\nu^2}{x}\cdot y=\lambda\cdot x\cdot y,\hfill
	\end{equation}
	то есть быть собственной функцией обобщённой задачи Штурма. Оператор $L$ рассматривается в области
	\begin{equation*}
		\hfill\mc{D}_{L}=\left\{y(x)|y\in\Cfn[{[0,R]}]{},\  y(R)=0,\ \begin{array}{rcl}
			\text{при }\nu=0&:&y\in\Cfn[{[0,R]}]{2},\\
			\text{при }\nu>0&:&y(0)=0,\ y\in\Cfn[{(0,R]}]{1},
		\end{array}\ \norm{Ly}_1<+\infty,\ \J[y]<+\infty\right\}_{\displaystyle.}\hfill
	\end{equation*}
	Было установлено, что в этой области оператор $L$ --- симметричен и 
	\begin{equation}\label{l9:eq:2}
		\hfill\big(Ly,y\big)=\J[y].\hfill
	\end{equation}
	Из~\eqref{l9:eq:1} следует, что в скалярном произведении $(u,v)_1=\smallint\limits_0^R x\cdot u(x)\cdot\overline{v}(x)\,dx$ собственные функции обобщённой задачи Штурма, отвечающие различным собственным значениям ортогональны, а собственные значения --- в силу~\eqref{l9:eq:2} --- неотрицательны, а при $\nu>0$ --- строго положительны, ибо в силу~\eqref{l9:eq:1},~\eqref{l9:eq:2}
	\begin{equation}\label{l9:eq:3}
		\hfill\big(Ly,y\big)=\J[y]=\int\limits_0^R\left(x\cdot y^{\prime2}+\frac{\nu^2}{x}\cdot y^2\right)\,dx=\lambda\cdot\norm{y}_1^2.\hfill
	\end{equation}   
	Отметим, что при $\nu=0$ равенство $\lambda=0$ возможно лишь при $y'=0$, то есть при $y=const$, а так как $y(R)=0$, то $y\equiv0$. Значит всегда $\lambda>0$. Однако, если бы было другое граничное условие при $x=R$: $y'(R)=0$, то функция $y\equiv const$ и $\lambda=0$ были бы собственной функцией и собственным значением обобщённой задачи Штурма. Мы позже вернёмся к этому случаю, а пока считаем $y(R)=0$ и значит $\lambda>0$. Так как оператор $L$ зависит от параметра $\nu$, то в~\eqref{l9:eq:1} мы будем писать $L=L(\nu)$, а решения~\eqref{l9:eq:1} и числа $\lambda$ занумеруем как $y_{\nu k}$ и $\lambda_{\nu k}$, $k=1,2,\ldots$, где $\nu$ --- фиксировано, а $k$ нумерует собственные значения и собственные функции обобщённой задачи Штурма. Таким образом~\eqref{l9:eq:1} запишется в виде 
	\begin{equation}\label{l9:eq:4}
		\hfill L(\nu)y_{\nu k}=\lambda_{\nu k}\cdot x\cdot y_{\nu k}.\hfill
	\end{equation}
	Введём в~\eqref{l9:eq:4} новую независимую переменную $\rho\eqdef\sqrt{\lambda_{\nu k}}\cdot x$ и положим $z(\rho)\equiv y_{\nu k}(x)$. Считаем $\nu$ и $k$ фиксированными и у функции $z_{\nu k}(\rho)=z(\rho)$ эти индексы писать не будем. Легко видеть, что для функции $z(\rho)$ мы получим следующее уравнение:
	\begin{equation}\label{l9:eq:5}
		\hfill\rho^2\cdot z''+\rho\cdot z'+\left(\rho^2-\nu^2\right)\cdot z=0,\hfill
	\end{equation}
	где в силу граничных условий и условий гладкости для функций $y_{\nu k}(x)$ мы должны потребовать
	\begin{equation*}
		\hfill z(\sqrt{\lambda}\cdot R)=0,\quad\text{при }\nu=0\ z(\rho)\in\Cfn[{[0,R]}]{2},\quad\text{при }\nu>0\ z(0)=0,\ z\in\Cfn[{[0,R]}]{2}.\hfill
	\end{equation*}

	Уравнение~\eqref{l9:eq:5} хорошо известно. Это уравнение Бесселя, его решения --- функции Бесселя $\nu$-ого порядка. Так как уравнение~\eqref{l9:eq:5} --- уравнение второго порядка --- то оно имеет два линейно независимых решения, называемые функциями Бесселя первого и второго рода. Но функция Бесселя второго рода $\nu$-ого порядка, или просто функция Бесселя $\nu$-ого порядка $J_\nu(\rho)$. Решение уравнения~\eqref{l9:eq:5} можно искать в виде ряда
	\begin{equation*}
		\hfill\rho^{\nu}\cdot\sum\limits_{k=0}^{\infty}C_k^{(\nu)}\cdot\rho^k.\hfill
	\end{equation*}  
	После подстановки в~\eqref{l9:eq:5} получим, что 
	\begin{equation}\label{l9:eq:6}
		\hfill J_{\nu}(\rho)=\rho^{\nu}\cdot\sum\limits_{k=0}^{\infty}C_{2\cdot k}^{(\nu)}\cdot\rho^{2\cdot k},\hfill
	\end{equation}
	где $\displaystyle C_{2\cdot k}^{(\nu)}=-C_{2\cdot k-2}^{(\nu)}\cdot\frac{1}{4\cdot k\cdot(k+1)}$,$\quad k=1,2,\ldots$
	
	\noindent Таким образом все коэффициенты выражаются через $C_0^{(\nu)}$, который свободен\footnote{Решение определено с точностью до множителя}.
	
	Итак, решение~\eqref{l9:eq:5} --- это $z(\rho)=J_{\nu}(\rho)$. Так как $y_{\nu k}(R)=0$, то $z(\sqrt{\lambda}\cdot R)=0=J_{\nu}\left(\sqrt{\lambda}\cdot R\right)$. Обозначим через $\mu_k^{(\nu)}$, $k=1,2,\ldots$ нули функции $J_{\nu}(\rho):\,J_{\nu}\left(\mu_k^{(\nu)}\right)=0$. Тогда $\sqrt{\lambda_{\nu k}}\cdot R=\mu_{k}^{(\nu)}$ и 
	\begin{equation}\label{l9:eq:7}
		\hfill\lambda_{\nu k}=\left(\frac{\mu_{k}^{(\nu)}}{R}\right)^2.\hfill
	\end{equation}
	Таким образом (так как $\rho=\sqrt{\lambda}\cdot x$)
	\begin{equation}\label{l9:eq:8}
		\hfill y_{\nu k}(x)=z(\rho)=J_{\nu}\left(\frac{\mu_{k}^{(\nu)}}{R}\cdot x\right)\footnotemark{}.\hfill
	\end{equation}\footnotetext{Теперь, когда $y_{\nu k}(x)=z$ найдены, надо проверить неравенства $\J[y_{\nu k}]<+\infty$, $\norm{Ly_{\nu k}}_1<+\infty$, которые должны выполняться в $\mc{D}_L$. Мы это сделаем позже.}

	Мы получили явный вид собственных функций $y_{\nu k}$ и собственных значений $\lambda_{\nu k}$ обобщённой задачи Штурма. Так как в выражения~\eqref{l9:eq:7},~\eqref{l9:eq:8} входят нули функции Бесселя $J_{\nu}(\rho)$, то проведём небольшое исследование тех точек $\mu_{k}^{(\nu)}$, в которых $J_{\nu}\left(\mu_{k}^{(\nu)}\right)=0$, то есть изучим --- хотя бы поверхностно, свойствва нулей функции Бесселя. Для этого воспользуемся известной асимптотикой функции Бесселя при больших $\rho$
	\begin{equation*}
		\hfill\sqrt{\rho}\cdot J_{\nu}(\rho)=a_{\nu}\cdot\cos(\rho-b\cdot\nu)+O\left(\frac{1}{\rho}\right),\hfill
	\end{equation*}
	где $a_{\nu}>0$ --- некоторое число, $\displaystyle b_{\nu}=\frac{\pi}{4}+\frac{\pi\cdot\nu}{2}$. Отсюда видно, что при $\rho=\rho_k=\pi\cdot k+b_{\nu}$ выполняется
	\begin{equation}\label{l9:eq:9}
		\hfill\sqrt{\rho_k}\cdot J_{\nu}(\rho_k)=(-1)^k\cdot a_{\nu}+O\left(\frac{1}{\rho_k}\right).\hfill
	\end{equation}
	Величина $J_{\nu}(\rho_k)$ и $J_{\nu}(\rho_{k+1})$ при больших $\rho$ имеют в силу~\eqref{l9:eq:9} разные знаки. И так как функция $J_{\nu}(\rho)$ непрерывна на отрезке $[\rho_k,\rho_{k+1}]$, на этом отрезке обязательно найдётся (и возможно не одна) точка, в которой $J_{\nu}(\rho)$ обращается в ноль. В тоже время, так как функция $J_{\nu}(\rho)$ --- аналитична на отрезке $[\rho_k,\rho_{k+1}]$ при $\rho_k>0$, то на этом отрезке \emph{не может быть бесконечного числа нулей}. Это относится вообще к любому отрезку $[\alpha,\beta]$ при $\alpha>0$. Что касается отрезка $[0,\alpha]$, то в силу~\eqref{l9:eq:6} при $0<\alpha<1$
	\begin{equation*}
		\hfill J_{\nu}(\rho)=\rho^{\nu}\cdot\Big(C_0^{(\nu)}+O(\rho)\Big),\quad\text{при}\quad\rho\neq0.\hfill
	\end{equation*}
	Следовательно, нули $\mu_{k}^{(\nu)}$ функции $J_{\nu}(\rho)$ могут накапливаться только на бесконечности. Поэтому собственные значения $\lambda_{\nu k}=\left(\mu_{k}^{(\nu)}/R\right)^2\to\infty$. Таким образом мы получили бесконечную последовательность при $k\to\infty$ собственных функций обобщённой задачи Штурма
	\begin{equation*}
		\hfill y_{\nu k}(x)=J_{\nu}\left(\frac{\mu_{k}^{(\nu)}}{R}\cdot x\right),\hfill
	\end{equation*}
	отвечающих собственным значениям 
	\begin{equation*}
		\hfill \lambda_{\nu k}=\left(\frac{\mu_{k}^{(\nu)}}{R}\right)^2\to\infty,\quad\text{при}\quad k\to\infty.\hfill
	\end{equation*}
	В силу свойств решений обобщённой задачи Штурма 
	\begin{equation*}
		\hfill\big(y_{\nu k},y_{\nu m}\big)_1=\left(J_{\nu}\left(\frac{\mu_{k}^{(\nu)}}{R}\cdot x\right),J_{\nu}\left(\frac{\mu_{m}^{(\nu)}}{R}\cdot x\right)\right),\quad k\neq m.\hfill
	\end{equation*}
	\begin{_teor}[Стеклова]
		Пусть
		\begin{equation*}
			\hfill d_k^{(\nu)}\eqdef\left(y,J_{\nu}\left(\frac{\mu_{k}^{(\nu)}}{R}\cdot x\right)\right)_1\Biggm/\norm{J_{\nu}\left(\frac{\mu_{k}^{(\nu)}}{R}\cdot x\right)}_1^2.\hfill
		\end{equation*} 
		Тогда 
		\begin{enumerateP1}
			\item\label{l9:Steclov:L}при $y\in\fLr[{[0,R];x}]$\quad $\displaystyle\norm{y-\sum\limits_{k=1}^{\infty}d_k^{(\nu)}\cdot J_{\nu}\left(\frac{\mu_{k}^{(\nu)}}{R}\cdot x\right)}_1\to0$, при $n\to\infty$,
			\item\label{l9:Steclov:D}при $y\in\mc{D}_L$\quad$\displaystyle\sup\limits_{x\in[a,b]}\left|y-\sum\limits_{k=1}^{\infty}d_k^{(\nu)}\cdot J_{\nu}\left(\frac{\mu_{k}^{(\nu)}}{R}\cdot x\right)\right|\to0$, при $n\to\infty$. 
		\end{enumerateP1} 
	\end{_teor}
	\noindent\ref{l9:Steclov:L} --- без доказательства, \ref{l9:Steclov:D} --- доказать при $y\in\mc{D}_L$. 
	
	Рассмотрим теперь случай граничного условия $y'(R)=0$ вместо $y(R)=0$. Обозначим соответствующие собственные функции и собственные значения обобщённой задачи Штурма через $\tilde{y}_{\nu k}(x)$ и через $\tilde{\lambda}_{\nu k}$. Вводим функцию $\tilde{z}(\rho)=\tilde{y}_{\nu k}(x)$, где $\rho=\sqrt{\tilde{\lambda}_{\nu k}}\cdot x$. Для функции $\tilde{z}(\rho)$ мы как и раньше получим уравнение Бесселя, выберем нужное нам решение $J_{\nu}(\rho)$, но теперь граничное условие при $x=R$, то есть при $\rho=\sqrt{\tilde{\lambda}}\cdot R$ будет $J_{\nu}'\left(\sqrt{\tilde{\lambda}}\cdot R\right)=0$. Обозначим нули функции $J_{\nu}'(\rho)$ через $\tilde{\mu}_k^{(\nu)}$ и тогда 
	\begin{equation*}
		\hfill\tilde{\lambda}_{\nu k}=\left(\frac{\tilde{\mu}_k^{(\nu)}}{R}\right)^2,\quad \tilde{y}_{\nu k}(x)=J_{\nu}\left(\frac{\tilde{\mu}_k^{(\nu)}}{R}\cdot x\right).\hfill
	\end{equation*}
	Чтобы убедится в существовании бесконечного числа нулей $\tilde{\mu}_k^{(\nu)}$ функции $J_{\nu}'(\rho)$ рассмотрим её асимптотику при $\rho\gg1$
	\begin{equation*}
		\hfill\sqrt{\rho}\cdot J_{\nu}'(\rho)=a_{\nu-1}\cdot\sin(\rho-b_{\nu})+O\left(\frac{1}{\rho}\right)\hfill
	\end{equation*}
	и действуем аналогично предыдущему. Дальнейшее рассмотрение обобщённой задачи Штурма с изменённым граничным условием не отличается от рассмотрения задачи с граничным условием $y(R)=0$.
	
	Теперь проверим, что функции $y_{\nu k}\in\mc{D}_L$. Нам осталось два утверждения $\J[y_{\nu k}]<+\infty$ и $\norm{Ly_{\nu k}}_1<+\infty$ ($\nu>0$). В силу~\eqref{l9:eq:6} и~\eqref{l9:eq:8} можно записать
	\begin{equation*}
		\hfill y_{\nu k}(x)=x^{\nu}\cdot\Phi(x),\hfill 
	\end{equation*}
	где, конечно, $\Phi(x)=\Phi_{\nu k}(x)$, но у нас $\nu$ и $k$ --- фиксированы и мы опускаем эти индексы. $\Phi(x)$ --- бесконечно дифференцируема на $[0,+\infty)$. Найдём $y_{\nu k}'$ и $y_{\nu k}''$
	\begin{equation*}
		\hfill y_{\nu k}'=\nu\cdot x^{\nu-1}\cdot\Phi(x)+x^{\nu}\cdot\Phi'(x),\quad y_{\nu k}''=\nu\cdot(\nu-1)\cdot\Phi(x)+2\cdot\nu\cdot x^{\nu-1}\cdot\Phi'(x)+x^{\nu}\cdot\Phi''(x).\hfill
	\end{equation*}
	Очевидно,
	\begin{equation*}
		\hfill x\cdot y_{\nu k}^{\prime2}\leqslant const\cdot\left(x^{2\cdot\nu}\cdot\Psi^{\prime2}(x)\cdot x+x^{2\cdot\nu}\cdot|\Psi(x)|\cdot|\Psi'(x)|+x^{2\cdot\nu-1}\cdot\Psi^2(x)\right).\hfill
	\end{equation*}
	Так как $\nu>0$, то
	\begin{equation*}
		\hfill\int\limits_0^R\left(x\cdot y_{\nu k}^{\prime2}+\frac{\nu^2}{x}\cdot y_{\nu k}\right)\,dx<+\infty\hfill
	\end{equation*}
	и 
	\begin{equation*}
		\hfill\norm{Ly_k}_1^2=\int\limits_0^R x\cdot\left[\der{}{x}\left(x\cdot y_{\nu k}'\right)+\frac{\nu}{x}\cdot y_{\nu k}\right]^2\,dx<+\infty.\hfill
	\end{equation*}
	то есть $y_{\nu k}\in\mc{D}_L$. 
	
	Но можно было проще: $Ly_{\nu k}=x\cdot\lambda_{\nu k}\cdot y_{\nu k}$ и значит
	\begin{equation*}
		\hfill\norm{Ly_{\nu k}}_1^2=\lambda_{\nu k}^2\cdot\norm{x\cdot y_{\nu k}}_1^2=\lambda_{\nu k}^2\cdot\int\limits_0^R x^3 y_{\nu k}^2<+\infty.\hfill
	\end{equation*} 
	\section[Функционалы, зависящие от функций двух переменных.]{Вариационные задачи для функционалов, зависящих от функций двух переменных.}
	\label{lecture9section2}
	Рассмотрим следующую задачу. Пусть дана цилиндрическая ёмкость с образующими параллельными оси $z$. В плоскости $x,\ y$ <<дно>> этой ёмкости образует какую-то область $G$. Сверху ёмкость не закрыта, граница цилиндрической поверхности сверху --- функция $f(x,y)$, $x,y\in\partial G$ ($\partial G$ --- граница $G$).
	
	

	\tikzset{every picture/.style={line width=0.75pt}} %set default line width to 0.75pt        
	
	\begin{tikzpicture}[x=0.75pt,y=0.75pt,yscale=-1,xscale=1]
		%uncomment if require: \path (0,250); %set diagram left start at 0, and has height of 250
		
		%Straight Lines [id:da7533107082747501] 
		\draw    (216.72,146.22) -- (414.78,146.22) ;
		\draw [shift={(416.78,146.22)}, rotate = 180] [color={rgb, 255:red, 0; green, 0; blue, 0 }  ][line width=0.75]    (10.93,-4.9) .. controls (6.95,-2.3) and (3.31,-0.67) .. (0,0) .. controls (3.31,0.67) and (6.95,2.3) .. (10.93,4.9)   ;
		%Straight Lines [id:da9386510837032684] 
		\draw    (216.72,146.22) -- (147.07,229.32) ;
		\draw [shift={(145.78,230.85)}, rotate = 309.97] [color={rgb, 255:red, 0; green, 0; blue, 0 }  ][line width=0.75]    (10.93,-4.9) .. controls (6.95,-2.3) and (3.31,-0.67) .. (0,0) .. controls (3.31,0.67) and (6.95,2.3) .. (10.93,4.9)   ;
		%Straight Lines [id:da22581012210637597] 
		\draw    (216.72,146.22) -- (216.72,4.85) ;
		\draw [shift={(216.72,2.85)}, rotate = 450] [color={rgb, 255:red, 0; green, 0; blue, 0 }  ][line width=0.75]    (10.93,-4.9) .. controls (6.95,-2.3) and (3.31,-0.67) .. (0,0) .. controls (3.31,0.67) and (6.95,2.3) .. (10.93,4.9)   ;
		%Shape: Polygon Curved [id:ds441803632619425] 
		\draw   (251.78,167.85) .. controls (263.78,160.85) and (292.78,157.85) .. (302.78,164.85) .. controls (312.78,171.85) and (334.22,192.15) .. (326,210) .. controls (317.78,227.85) and (243.22,222.15) .. (236,210) .. controls (228.78,197.85) and (239.78,174.85) .. (251.78,167.85) -- cycle ;
		%Shape: Polygon Curved [id:ds2324963963699278] 
		\draw   (249.78,44.85) .. controls (261.78,37.85) and (291.78,34.85) .. (300.78,41.85) .. controls (309.78,48.85) and (332.22,69.15) .. (324,87) .. controls (315.78,104.85) and (241.22,99.15) .. (234,87) .. controls (226.78,74.85) and (237.78,51.85) .. (249.78,44.85) -- cycle ;
		%Straight Lines [id:da6083220832384053] 
		\draw  [dash pattern={on 4.5pt off 4.5pt}]  (232,80) -- (234,203) ;
		%Straight Lines [id:da7239634925795717] 
		\draw  [dash pattern={on 4.5pt off 4.5pt}]  (326,80) -- (328,203) ;
		
		% Text Node
		\draw (213.72,6.25) node [anchor=north east] [inner sep=0.75pt]    {$z$};
		% Text Node
		\draw (143.78,227.45) node [anchor=south east] [inner sep=0.75pt]    {$x$};
		% Text Node
		\draw (418.78,145.62) node [anchor=north west][inner sep=0.75pt]    {$y$};
		% Text Node
		\draw (275,182.4) node [anchor=north west][inner sep=0.75pt]    {$G$};
		
		
	\end{tikzpicture}
	
	\noindent\textbf{Задача: }закрыть цилиндр крышкой наименьшей площади. Если обозначить закрывающую поверхность через $z(x,y)$, то площадь крышки
	\begin{equation*}
		\hfill S[z]=\iint\limits_{G}\sqrt{1+z_x^2+z_y^2}\,dxdy,\hfill
	\end{equation*}
	и мы должны минимизировать функционал $S[z]$ в классе функций $z(x,y)$, $z\Big|_{\partial G}=f(x,y)$, $z\in\Cfn[]{1}(G)$.
	
	Перейдём теперь к общему случаю. Вводим обозначения $p\eqdef\displaystyle\pder{z}{x}$, $q\eqdef\displaystyle\pder{z}{y}$. Рассмотрим функционал
	\begin{equation*}
		\hfill\J[z]=\iint\limits_{G}F(x,y,z,p,q)\,dxdy,\hfill
	\end{equation*}
	где $G$ --- некоторая область в плоскости $x,\ y$, $F\in\Cfn{2}$ при $(x,y)\in G$, $|z|<M$, $\forall p,\ q$; здесь $M$ --- какая-то большая константа. Пусть $P\eqdef(x,y)$. И пусть
	\begin{equation*}
		\hfill\K\eqdef\left\{z(x,y)|z\in\Cfn{1}(\overline{G}),\ z\Big|_{\partial G}=f(P),\ |z|<M\right\}.\hfill
	\end{equation*}
	Мы будем рассматривать задачу на $\displaystyle\min\limits_{z\in\K}\,\J[z]$. Как обычно предполагаем, что минимайзер существует и обладает повышенной гладкостью. Пусть $z=z(x,y)$ --- минимайзер в рассматриваемой задаче.
	\begin{_def}
		Функцию $\eta(x,y)$ назовём \textbf{допустимым изменением}, если $\tilde{z}\eqdef z+t\cdot\eta\in\K$ при $|t|\ll1$.
	\end{_def}  
	Легко видеть, что отсюда вытекают такие ограничения на $\eta$: $\eta\Big|_{\partial G}=0$, $\eta\in\Cfn{1}(\overline{G})$. Далее проводим стандартные рассуждения. Полагаем $\phi(t)=\J[z+t\cdot\eta]$.
	\begin{equation*}
		\J[z+t\cdot\eta]\geqslant\J[z]\quad\Rightarrow\quad\phi(t)\geqslant\phi(0);
	\end{equation*}
	\begin{equation*}
		\J[z+t\cdot\eta]-\J[z]=\delta\J+\ldots,\qquad\phi(t)-\phi(0)=\phi'(0)\cdot t+\ldots;
	\end{equation*}
	\begin{equation*}
		\hfill\delta\J=t\cdot\phi'(0)=t\cdot\left.\der{}{t}\J[z+t\cdot\eta]\right|_{t=0}\quad\text{--- первая вариация.}\hfill
	\end{equation*}
	Так как $\phi(0)$ --- минимальное значение $\phi(t)$ при $|t|\ll1$, то $\phi'(0)=0$ (функция $\phi(t)\in\Cfn{1}$!), то есть $\delta\J=0$ --- необходимое условие минимума (для максимума --- тоже). Вычислим первую вариацию функционала 
	\begin{equation*}
		\hfill\J[z+t\cdot\eta]=\iint\limits_{G}\overbrace{F(x,y,\underbrace{z+t\cdot\eta}_{\tilde{z}},\underbrace{p+t\cdot\eta_x}_{\tilde{z}_x},\underbrace{q+t\cdot\eta_y}_{\tilde{z}_y})}^{\widetilde{F}}\,dxdy.\hfill
	\end{equation*}
	Имеем (предполагая только $z\in\Cfn{2}$, $\eta\in\Cfn{1}$.)
	\begin{multline*}
		\delta\J=\left.t\cdot\der{}{t}\iint\limits_{G}\widetilde{F}\,dxdy\right|_{t=0}=\left.t\cdot\iint\limits_{G}\left(\widetilde{F}_{\tilde{z}}\cdot\eta+\widetilde{F}_{\tilde{z}_x}\cdot\eta_x+\widetilde{F}_{\tilde{z}_y}\cdot\eta_y\right)\,dxdy\right|_{t=0}=\\=t\cdot\iint\limits_{G}\left({F}_{z}\cdot\eta+{F}_{z_x}\cdot\eta_x+{F}_{z_y}\cdot\eta_y\right)\,dxdy.
	\end{multline*}
	Преобразуем полученное выражение так, чтобы оно не содержало производных $\eta_x$, $\eta_y$. Имеем выделяя дивергентную форму 
	\begin{equation*}
		\iint\limits_{G}\left(F_p\cdot\eta_x+F_q\cdot\eta_y\right)\,dxdy=\iint\limits_{G}\left(\der{}{x}(F_p\cdot\eta)+\der{}{y}(F_q\cdot\eta)\right)\,dxdy-\iint\limits_{G}\left(\der{}{x}F_p+\der{}{y}F_q\right)\cdot\eta\,dxdy.
	\end{equation*}
	Таким образом, первая вариация
	\begin{equation*}
		\hfill\delta\J=t\left[\iint\limits_{G}\left(F_z-\der{}{x}F_p-\der{}{y}F_q\right)\cdot\eta\,dxdy+\iint\limits_{G}\left(\der{}{x}(F_p\cdot\eta)+\der{}{y}(F_q\cdot\eta)\right)\,dxdy\right]_{\displaystyle.}\hfill
	\end{equation*}
	Введём в рассмотрение вектор $\bm{\Phi}=(F_p\cdot\eta,F_q\cdot\eta)$. Его компоненты обладают гладкостью $\Cfn{1}(G)$ и $\Cfn{}(\overline{G})$. Поэтому можно применить формулу Остроградского--Гаусса, согласно которой
	\begin{center}
		\fbox{\parbox{0.32\textwidth}{$\displaystyle\iint\limits_{G}\Div\bm{\Phi}\,dxdy=\int\limits_{\partial G}\big(\bm{\Phi},\bm{n}\big)_{\R^2}\,dl,$}}
	\end{center}
	где $\bm{n}=(\cos\alpha,\cos\beta)$ --- единичная внешняя нормаль к границе $\partial G$.
	
	\noindent Используя эту формулу получаем окончательное выражение для первой вариации
	\begin{equation}\label{l9:eq:10}
		\hfill\delta\J=t\cdot\iint\limits_{G}\left(F_z-\der{}{x}F_p-\der{}{y}F_q\right)\cdot\eta\,dxdy+t\cdot\int\limits_{\partial G}(F_p\cdot\cos\alpha+F_q\cdot\cos\beta)\cdot\eta\,dl.\hfill
	\end{equation}
	Заметим, что это выражение даёт главную часто приращения $\J[z+t\cdot\eta]=\J[z]$ не зависимо ни от каких свойств $z$ и $\eta$ кроме гладкости. Мы этим будем пользоваться.
	
	Теперь возвращаемся к нашей вариационной задаче. Если $z$ --- минимайзер и $\eta$ --- допустимое изменение, то $\eta\Big|_{\partial G}\equiv0$ и $\delta\J=0$, то есть
	\begin{equation}\label{l9:eq:11}
		\hfill\iint\limits_{G}\left(F_z-\der{}{x}F_p-\der{}{y}F_q\right)\cdot\eta\,dxdy=0,\quad\forall\eta\text{ --- допустимого.}\hfill
	\end{equation}
	Пусть $\displaystyle\Psi\equiv F_z-\der{}{x}F_p-\der{}{y}F_q$. Тогда~\eqref{l9:eq:11} означает, что 
	\begin{equation}\label{l9:eq:12}
		\hfill\iint\limits_{G}\Psi(x,y)\cdot\eta\,dxdy=0\quad\forall\eta\text{ --- допустимого.}\hfill
	\end{equation}
	Из равенства~\eqref{l9:eq:12} вытекает, что $\Psi(x,y)\equiv0$, $x,y\in G$ (Обобщение леммы Лагранжа на плоскость). Действительно, пусть в некоторой точке $x_0,\ y_0$ $\Psi(x_0,y_0)>0$. Так как функция $\Psi(x,y)$ --- непрерывная, то можно указать такой квадрат $A$ со стороной $2\cdot\delta$: $|x-x_0|\leqslant\delta$, $|y-y_0|\leqslant\delta$, что $\Psi(x,y)>0$, $x,\ y\in A$.
	
	
	\tikzset{every picture/.style={line width=0.75pt}} %set default line width to 0.75pt        
	
	\begin{tikzpicture}[x=0.75pt,y=0.75pt,yscale=-1,xscale=1]
		%uncomment if require: \path (0,199); %set diagram left start at 0, and has height of 199
		
		%Shape: Polygon Curved [id:ds8647161812959627] 
		\draw   (71.59,37) .. controls (99.39,19.82) and (166.58,12.45) .. (189.74,29.64) .. controls (212.91,46.82) and (262.56,96.65) .. (243.53,140.48) .. controls (224.49,184.32) and (51.75,170.3) .. (35.03,140.48) .. controls (18.31,110.66) and (43.79,54.19) .. (71.59,37) -- cycle ;
		%Shape: Circle [id:dp30365118515163947] 
		\draw  [fill={rgb, 255:red, 0; green, 0; blue, 0 }  ,fill opacity=1 ] (146,104.93) .. controls (146,103.78) and (146.93,102.85) .. (148.07,102.85) .. controls (149.22,102.85) and (150.15,103.78) .. (150.15,104.93) .. controls (150.15,106.07) and (149.22,107) .. (148.07,107) .. controls (146.93,107) and (146,106.07) .. (146,104.93) -- cycle ;
		%Shape: Square [id:dp9936568887522785] 
		\draw   (125.52,82.38) -- (170.62,82.38) -- (170.62,127.48) -- (125.52,127.48) -- cycle ;
		%Straight Lines [id:da11258115446689132] 
		\draw    (40.78,34.85) -- (120.86,88.27) ;
		\draw [shift={(122.52,89.38)}, rotate = 213.7] [color={rgb, 255:red, 0; green, 0; blue, 0 }  ][line width=0.75]    (10.93,-3.29) .. controls (6.95,-1.4) and (3.31,-0.3) .. (0,0) .. controls (3.31,0.3) and (6.95,1.4) .. (10.93,3.29)   ;
		
		% Text Node
		\draw (242,153.4) node [anchor=north west][inner sep=0.75pt]    {$G$};
		% Text Node
		\draw (132,107.4) node [anchor=north west][inner sep=0.75pt]  [font=\scriptsize]  {$x_{0} ,\ y_{0}$};
		% Text Node
		\draw (38.78,31.45) node [anchor=south east] [inner sep=0.75pt]    {$A$};
		
		
	\end{tikzpicture}
	
	\noindent Определим функцию $\eta(x,y)$ равенствами 
	\begin{equation*}
		\eta(x,y)=\begin{cases}
			\left[(x-x_0+\delta)^2\cdot(x-x_0-\delta)^2\cdot(y-y_0+\delta)^2\cdot(y-y_0-\delta)^2\right]& x,y\in A,\\
			0&x,y\notin A.
		\end{cases}
	\end{equation*}
	Ясно, что по определению $\eta\Big|_{\partial G}=0$ и $\displaystyle\pder{\eta}{x}=\pder{\eta}{y}=0$, $x, y\in\partial G$. Поэтому $\eta\in\Cfn{1}(\overline{G})$. Подставляя в~\eqref{l9:eq:11} получим
	\begin{equation*}
		\hfill\iint\limits_{A}\Psi(x,y)\cdot\eta(x,y)\,dxdy=0,\hfill
	\end{equation*}
	что невозможно, ибо в области $A$ функции $\Psi(x,y)$ и $\eta(x,y)$ --- положительны (кроме границы). Значит, не может существовать такая точка $x_0,\ y_0$ в которой $\Psi(x_0, y_0)>0$. Аналогично убеждаемся, что не может быть точки $x_0,\ y_0$, в которой $\Psi(x_0,y_0)<0$. Значит $\Psi(x,y)\equiv0$, то есть
	\begin{equation*}
		\hfill F_z-\der{}{x}F_p-\der{}{y}F_q\equiv0\hfill
	\end{equation*} 
	если в функцию $F(x,y,z,p,q)$ подставили минимайзер, то есть для нахождения минимайзера надо решить уравнение
	\begin{equation}\label{l9:eq:13}
		\hfill F_z-\der{}{x}F_p-\der{}{y}F_q=0.\hfill
	\end{equation}
	Это --- \textbf{уравнение Остроградского}. Мы должны решать его в классе функций $z(x,y)\in\Cfn{2}(G)$, $z\Big|_{\partial G}=f(x,y)$.
	
	Уравнение~\eqref{l9:eq:13} --- это уравнение в частных производных второго порядка. Проведём дифференцирование
	\begin{equation*}
		F_z-F_{px}-F_{pz}\cdot\pder{z}{x}-F_{pp}\cdot\pdder{z}{x}-F_{pq}\cdot\pder{^2z}{x\partial y}-F_{qy}-F_{qz}\cdot\pder{z}{y}-F_{qp}\cdot\pder{^2z}{x\partial y}-F_{qq}\cdot\pdder{z}{y}=0,
	\end{equation*}
	то есть
	\begin{equation}\label{l9:eq:14}
		\hfill-F_{pp}\cdot\pdder{z}{x}-2\cdot F_{pq}\cdot\pder{^2z}{x\partial y}-F_{qq}\cdot\pdder{z}{y}=W(x,y,z,p,q).\hfill
	\end{equation}
	
	Рассмотрим пример. Интеграл
	\begin{equation*}
		\hfill\J[z]=\iint\limits_{G}\left[\left(\pder{z}{x}\right)^2+\left(\pder{z}{y}\right)^2\right]\,dxdy=\iint\limits_{G}\left(p^2+q^2\right)\,dxdy\hfill
	\end{equation*}
	называется интегралом Дирихле. Уравнение Остроградского для него 
	\begin{equation*}
		\hfill-\der{}{x}(2\cdot z_x)-\der{}{y}(2\cdot z_y)=0,\hfill
	\end{equation*}
	то есть
	\begin{equation*}
		\hfill\pdder{z}{x}+\pdder{z}{y}=0\text{ --- уравнение Лапласа},\hfill
	\end{equation*}
	Оператор $\displaystyle\Delta\eqdef\pdder{}{x}+\pdder{}{y}$ --- оператор Лапласа.
	
	\noindent\underline{Обобщение}. Рассмотрим функционал $\J[z]$ для функции $z=z(x_1,\ldots,x_n)$ зависящей от $n$ переменных. Пусть $\displaystyle p_i\eqdef\pder{z}{x_i}$, $i=\overline{1,n}$, $\Omega=\{x_1,\ldots,x_n\}$ --- некоторая область.
	\begin{equation*}
		\hfill\J[z]=\idotsint\limits_{\Omega}F(x_1,\ldots,x_n,z,p_1,\ldots,p_n)\,d\Omega.\hfill
	\end{equation*}
	$z\Big|_{\partial\Omega}=f(x_1,\ldots,x_n)$. Уравнение Остроградского для этого функционала выводится совершенно аналогично. В результате мы получим 
	\begin{equation*}
		\hfill F_z-\der{}{x_1}F_{p_1}-\der{}{x_2}F_{p_2}-\ldots-\der{}{x_n}F_{p_n}=0.\hfill
	\end{equation*}
	\section{Вариационные задачи со свободной границей.}
	\label{lecture9section3}
	Возвращаемся к исходному функционалу
	\begin{equation*}
		\hfill\J[z]=\iint\limits_{G}F(x,y,z,p,q)\,dxdy,\hfill
	\end{equation*}
	но класс допустимых функций в задаче на $\min\,\J[z]$ определим по-другому. Пусть кривая $\Gamma\eqdef\partial G$. Разобьём $\Gamma$ произвольным образом на две части $\Gamma=\Gamma_1\cup \Gamma_2$, причём 
	\begin{enumerate}
		\item одна из частей может отсутствовать,
		\item кривые $\Gamma_1$ и $\Gamma_2$ могут состоять каждая из отдельных кусков (см.~картинку, например; $\Gamma_1$ выделена жирным).
	\end{enumerate} 
	
	\tikzset{every picture/.style={line width=0.75pt}} %set default line width to 0.75pt        
	
	\begin{tikzpicture}[x=0.75pt,y=0.75pt,yscale=-1,xscale=1]
		%uncomment if require: \path (0,187); %set diagram left start at 0, and has height of 187
		
		%Shape: Polygon Curved [id:ds8076294015434553] 
		\draw   (63.89,42.58) .. controls (87.24,27.83) and (143.68,21.5) .. (163.14,36.26) .. controls (182.6,51.02) and (224.31,93.79) .. (208.32,131.42) .. controls (192.33,169.06) and (47.22,157.03) .. (33.18,131.42) .. controls (19.13,105.82) and (40.54,57.34) .. (63.89,42.58) -- cycle ;
		%Curve Lines [id:da5463349166005875] 
		\draw [line width=2.25]    (163.14,36.26) .. controls (181.89,49.86) and (220.89,91.86) .. (208.32,131.42) ;
		%Curve Lines [id:da42261145901201513] 
		\draw [line width=2.25]    (33.18,131.42) .. controls (20.89,101.86) and (39.89,62.86) .. (63.89,42.58) ;
		
		% Text Node
		\draw (111,85.4) node [anchor=north west][inner sep=0.75pt]    {$G$};
		% Text Node
		\draw (203,61.4) node [anchor=north west][inner sep=0.75pt]    {$\Gamma _{1}$};
		% Text Node
		\draw (25,43.4) node [anchor=north west][inner sep=0.75pt]    {$\Gamma _{1}$};
		% Text Node
		\draw (85,156.4) node [anchor=north west][inner sep=0.75pt]    {$\Gamma _{2}$};
		% Text Node
		\draw (90,7.4) node [anchor=north west][inner sep=0.75pt]    {$\Gamma _{2}$};
		
		
	\end{tikzpicture}
	
	Причина подобного разбиения --- необходимость описать ситуацию, когда часть границы --- скажем $\Gamma_2$ --- свободна, а на $\Gamma_1$ задано граничное условие (до сих пор $\Gamma_2=\varnothing$).
	
	Итак, определим класс допустимых функций
	\begin{equation*}
		\hfill\K=\left\{z(x,y)|z\in\Cfn{1}(\overline{G}),\ z\Big|_{\Gamma_1}=f(x,y),\ \Gamma_2\neq\varnothing,\ |z|<M\right\}.\hfill
	\end{equation*}  
	Тогда, чтобы $z+t\cdot\eta\in\K$ нам необходимо требовать, чтобы $\eta(x,y)\equiv0$ не на всей кривой $\Gamma=\partial G$, а только на $\Gamma_1$. Достаточно условия: $\eta\Big|_{\Gamma_1}\equiv0$, а $\eta\Big|_{\Gamma_2}$ --- произвольна. Разумеется, требование $\eta(x,y)\in\Cfn{1}(\overline{G})$ --- сохраняется. Если $z$ --- минимайзер, то $\delta\J=0$, а формула для $\delta\J$ --- это~\eqref{l9:eq:10}. Значит
	\begin{equation*}
		\hfill\delta\J=t\left[\iint\limits_{G}\left(F_z-\der{}{x}F_p-\der{}{y}F_q\right)\cdot\eta\,dxdy+\int\limits_{\Gamma}\left(F_p\cdot\cos\alpha+F_q\cdot\cos\beta\right)\cdot\eta\,dl\right]=0.\hfill
	\end{equation*}
	Взяв $\eta\Big|_{\Gamma}\equiv0$ (это не запрещено) мы получаем, что
	\begin{equation*}
		\hfill\iint\limits_{G}\left(F_z-\der{}{x}F_p-\der{}{y}F_q\right)\cdot\eta\,dxdy=0,\hfill
	\end{equation*}
	откуда как и раньше выводим уравнение Остроградского
	\begin{equation*}
		\hfill F_z-\der{}{x}F_p-\der{}{y}F_q=0.\hfill
	\end{equation*}
	Значит равенство $\delta\J=0$ сводится к равенству
	\begin{equation*}
		\hfill\int\limits_{\Gamma}(F_p\cdot\cos\alpha+F_q\cdot\cos\beta)\cdot\eta\,dl=0,\quad\forall\eta\text{ --- допустимое.}\hfill
	\end{equation*}
	Но так как на $\Gamma_1$ $\eta\Big|_{\Gamma_1}\equiv0$, а $\Gamma=\Gamma_1\cup\Gamma_2$, то мы получаем, что
	\begin{equation}\label{l9:eq:15}
		\hfill\int\limits_{\Gamma_2}(F_p\cdot\cos\alpha+F_q\cdot\cos\beta)\cdot\eta\,dl=0.\hfill
	\end{equation}
	В силу произвольности функции $\eta$ на $\Gamma_2$ мы, обобщая лемму Лагранжа на случай кривой, получим, что
	\begin{equation}\label{l9:eq:16}
		\hfill F_p\cdot\cos\alpha+F_q\cdot\cos\beta=0,\quad(x,y)\in\Gamma_2.\hfill
	\end{equation}
	Это --- естественное граничное условие для свободной части границы. Если $\Gamma_1=\varnothing$, то условие~\eqref{l9:eq:16} должно выполняться на всей границе $\Gamma=\partial G$. Таким образом уравнение Остроградского надо решать с граничными условиями: на $\Gamma_2$ ---~\eqref{l9:eq:16}, на $\Gamma_1$ $z\Big|_{\Gamma_1}=f(x,y)$.
	
	Приведём пример. Пусть $F=z_x^2+z_y^2$, то есть мы рассматриваем интеграл Дирихле. Тогда \begin{equation*}
		F_p\cdot\cos\alpha+F_q\cdot\cos\beta=2\cdot(z_x\cdot\cos\alpha+z_y\cdot\cos\beta)=2\cdot\big(\nabla z,\bm{n}\big)=2\cdot\pder{z}{n}
	\end{equation*}
	и естественное граничное условие на свободной части границы $\displaystyle\left.\pder{z}{n}\right|_{\Gamma_2}=0$
	
	
	
	\tikzset{every picture/.style={line width=0.75pt}} %set default line width to 0.75pt        
	
	\begin{tikzpicture}[x=0.75pt,y=0.75pt,yscale=-1,xscale=1]
		%uncomment if require: \path (0,222); %set diagram left start at 0, and has height of 222
		
		%Shape: Axis 2D [id:dp7578165526074085] 
		\draw  (33,173.57) -- (297.78,173.57)(59.48,27) -- (59.48,189.85) (290.78,168.57) -- (297.78,173.57) -- (290.78,178.57) (54.48,34) -- (59.48,27) -- (64.48,34)  ;
		%Shape: Rectangle [id:dp16296921235920236] 
		\draw   (83,78) -- (213.78,78) -- (213.78,139.85) -- (83,139.85) -- cycle ;
		%Straight Lines [id:da6454278529068038] 
		\draw  [dash pattern={on 4.5pt off 4.5pt}]  (83,139.85) -- (82.78,172.85) ;
		%Straight Lines [id:da14183687134341394] 
		\draw  [dash pattern={on 4.5pt off 4.5pt}]  (213.78,139.85) -- (213.57,172.85) ;
		
		% Text Node
		\draw (299,172.4) node [anchor=north west][inner sep=0.75pt]    {$x$};
		% Text Node
		\draw (84.78,181.25) node [anchor=north west][inner sep=0.75pt]    {$a$};
		% Text Node
		\draw (215.57,176.25) node [anchor=north west][inner sep=0.75pt]    {$b$};
		% Text Node
		\draw (42,31.4) node [anchor=north west][inner sep=0.75pt]    {$y$};
		% Text Node
		\draw (85,101.4) node [anchor=north west][inner sep=0.75pt]    {$\Gamma _{2}$};
		% Text Node
		\draw (194,101.4) node [anchor=north west][inner sep=0.75pt]    {$\Gamma _{2}$};
		% Text Node
		\draw (303,101.4) node [anchor=north west][inner sep=0.75pt]    {$\displaystyle\left. \frac{\partial z}{\partial x}\right| _{x=b} =0,$};
		% Text Node
		\draw (400,101.4) node [anchor=north west][inner sep=0.75pt]    {$\displaystyle\left. -\frac{\partial z}{\partial x}\right| _{x=a} =0.$};
		
	\end{tikzpicture}
\end{document}
